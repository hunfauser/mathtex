\documentclass[a4paper,12pt]{article}

\usepackage{cmap}					% поиск в PDF
\usepackage{mathtext} 				% русские буквы в формулах
\usepackage[T2A]{fontenc}			% кодировка
\usepackage[utf8]{inputenc}			% кодировка исходного текста
\usepackage[english,russian]{babel}	% локализация и переносы

% Дополнительная работа с математикой
\usepackage{amsmath,amsfonts,amssymb,amsthm,mathtools} % AMS
\usepackage{icomma}                                    % "Умная" запятая

%% Шрифты
\usepackage{euscript} % Шрифт Евклид
\usepackage{mathrsfs} % Матшрифт

%% Перенос знаков в формулах (по Львовскому)
\newcommand*{\hm}[1]{#1\nobreak\discretionary{}
{\hbox{$\mathsurround=0pt #1$}}{}}

%%% Работа с картинками
\usepackage{graphicx} % Для вставки рисунков
\usepackage{wrapfig}  % Обтекание рисунков текстом

%%% Работа с таблицами
\usepackage{array,tabularx,tabulary,booktabs} % Дополнительная работа с таблицами
\usepackage{longtable}                        % Длинные таблицы
\usepackage{multirow}                         % Слияние строк в таблице

\usepackage{pgf,tikz} % Работа с графикой
\usepackage{pgfplots}
\usepackage{pgfplotstable}

\usepackage[]{geometry}

\usetikzlibrary{arrows}

%%% Заголовок
\author{Hun Fauser}
\title{Геометрия}
\date{\today}

\begin{document} % Конец преамбулы, начало текста.

\maketitle

\newpage
\section{Площадь круга}

\begin{figure}[h]
\begin{center}
	\begin{tikzpicture}
		\draw (0, 0) circle (2.5);   % Нарисовать круг
		\draw (0, -2.5) -- (0, 2.5); % Нарисовать вертикальную линию
		\draw (0, 0) -- (2.5, 0);    % Нарисовать горизонтальную линию
	\begin{scriptsize}
		\draw (-0.25, -0.25) node {$D$};
		\draw (1, -0.25) node {$r$};
	\end{scriptsize}
	\end{tikzpicture}
	\caption{Зная диаметр или радиус круга, можно найти его площадь.}
\end{center}
\end{figure}

\subsection{Обозначения}

r = радиус круга \\
R = диаметр \\
$\pi \approx 3.14$

\subsection{Площадь круга}

\[ S = \pi r^2 = \frac{\pi}{4}D^2 \]

\subsection{Периметр круга}

\[ p = 2\pi r = \pi D \]

\newpage
\section{Площадь прямоугольника}

\begin{figure}[h]
	\begin{center}
\begin{tikzpicture}[auto]
	\draw (0, 0) -- (0, 3); %(A, B)
	\draw (0, 0) -- (5, 0); %(B, C)
	\draw (5, 0) -- (5, 3); %(C, D)
	\draw (5, 3) -- (0, 3); %(A, D)
\begin{scriptsize}
\draw (-0.25, 3.25) node {$A$};
\draw (-0.25, -0.25) node {$B$};
\draw (5.25, -0.25) node {$C$};
\draw (5.25, 3.25) node {$D$};

\draw (-0.25, 1.5) node {$a$};
\draw (2.5, -0.25) node {$b$};
\draw (5.25, 1.5) node {$c$};
\draw (2.5, 3.25) node {$d$};
\end{scriptsize}
\end{tikzpicture}
\end{center}
\end{figure}

\subsection{Обозначения}

b, d = длина прямоугольника \\
a, c = ширина прямоугольника

\subsection{Площадь прямоугольника}

\[S = ab \]

\newpage
\section{Площадь эллипса}

\begin{figure}[h]
\begin{center}
	\begin{tikzpicture}
		\draw (0, 0) ellipse (3cm and 2cm); % Нарисовать эллипс
		\draw (0, 0) -- (0, 2);     	   % Нарисовать вертикальную линию
		\draw (0, 0) -- (3, 0);    		   % Нарисовать горизонтальную линию
		\draw[dashed] (-3, 0) -- (0, 0);   % Нарисовать горизонтальную пунктирную линию
		\draw[dashed] (0, 0) -- (0, -2);   % Нарисовать вертикальную пунктирную линию
	\begin{scriptsize}
		\draw (1.5, -0.25) node {$R$};
		\draw (-0.25, 1) node {$r$};
	\end{scriptsize}
	\end{tikzpicture}
\end{center}
\end{figure}

\subsection{Обозначения}

R = большая полуось \\
r = малая полуось

\subsection{Площадь прямоугольника}

\[ S = \pi Rr \]

\newpage
\section{Формула площади равнобедренной трапеции через стороны и угол}

\begin{figure}[h]
\begin{center}
	\begin{tikzpicture}
		\draw (0, 0) -- (5, 0);     %(B, C)
		\draw (0, 0) -- (1, 2.5);   %(B, A)
		\draw (1, 2.5) -- (4, 2.5); %(A, D)
		\draw (4, 2.5) -- (5, 0);   %(D, C)
	\begin{scriptsize}		
		\draw[fill=black] (0, 0) circle (1.5pt);
		\draw[fill=black] (5, 0) circle (1.5pt);
		\draw[fill=black] (1, 2.5) circle (1.5pt);
		\draw[fill=black] (4, 2.5) circle (1.5pt);
		
		\draw (2.5, -0.25) node {$a$};
		\draw (2.5, 2.75) node {$b$};
		\draw (0.25, 1.25) node {$c$};
		\draw (4.75, 1.25) node {$c$};
		\draw (0.7, 0.5) node {$\alpha$};
		
		\draw (0.4, 1.25) -- (0.6, 1.15);  % слева
		\draw (0.35, 1.20) -- (0.55, 1.1); % слева
		\draw (4.6, 1.25) -- (4.4, 1.15);  % справа
		\draw (4.65, 1.2) -- (4.45, 1.1);  % справа
		
		\draw [shift={(1, 2.5)}] (0, 0) -- (-111.5 : 0.5) arc (-111.5 : 0 : 0.5) -- cycle; % Угол А
		\draw [shift={(4, 2.5)}] (0, 0) -- (180 : 0.5) arc (185 : 290 : 0.5) -- cycle;     % Угол D
		
		\draw [shift={(5,0)}] (111 : 0.65) arc (112 : 180 : 0.65); % Угол C
		\draw [shift={(5,0)}] (112 : 0.55) arc (112 : 180 : 0.55); % Угол C
		\draw [shift={(0,0)}] (0 : 0.65) arc (0 : 68 : 0.65);      % Угол D
		\draw [shift={(0,0)}] (0 : 0.55) arc (0 : 68 : 0.55);      % Угол D
	\end{scriptsize}
	\end{tikzpicture}
\end{center}
\end{figure}

\subsection{Обозначения}

а - нижнее основание \\
b - верхнее основание \\
с - равные боковые стороны \\
$\alpha$ - угол при нижнем основании

\subsection{Формула площади равнобедренной трапеции через стороны}

\[ S = \frac{a + b}{2}\sqrt{c^2 - \frac{(a - b)^2}{4}} \]

\subsection{Формула площади равнобедренной трапеции через стороны и угол}

\begin{align*}
S = \frac{a^2 - b^2}{4}\tg\alpha \\
S = c \times \sin\alpha(a - \cos\alpha) \\
S = c \times \sin\alpha(b + \cos\alpha)
\end{align*}

\newpage
\section{Формула площади равнобедренной трапеции через стороны и угол}

\begin{figure}[h]
\begin{center}
	\begin{tikzpicture}

	\draw [shift={(-1, 1.45)}] (0, 0) -- (-109 : 0.25) arc (-109 : 0 : 0.25) -- cycle; % левый верхний угол
	\draw [shift={(1, 1.45)}] (0, 0) -- (180 : 0.25) arc (180 : 289 : 0.25) -- cycle;  % правый верхний угол

	\draw (0.25, -1.45) -- (0.25, -1.25) -- (0, -1.25) -- (0, -1.45) -- cycle; % угол 90deg

	\draw(0, 0) circle (1.45cm); % круг

	\draw (-1, 1.45) -- (1, 1.45);       % верхняя линия
	\draw (1, 1.45) -- (2.1, -1.45);     % линия справа
	\draw (2.1, -1.45) -- (-2.1, -1.45); % нижняя линия
	\draw (-2.1, -1.45) -- (-1, 1.45);   % линия слева

	\draw [shift={(2.1, -1.45)}] (109 : 0.35) arc (109 : 180 : 0.35); % нижний правый угол
	\draw [shift={(2.1,-1.45)}] (109 : 0.3) arc (109 : 180 : 0.3);    % нижний правый угол

	\draw [shift={(-2.1, -1.45)}] (0 : 0.35) arc (0 : 70 : 0.35);     % нижний левый угол
	\draw [shift={(-2.1, -1.45)}] (0 : 0.3) arc (0 : 70 : 0.3);       % нижний левый угол

	\draw (0, 1.45) -- (0, -1.45); % диаметр вписанной окружности
	\draw (0, 0) -- (1, 1.05);     % радиус вписанной окружности

	\draw (-1.7, 0) -- (-1.4, -0.08); % слева
	\draw (-1.68, 0.08) -- (-1.4, 0); % слева

	\draw (1.4, -0.08) -- (1.7, 0); % справа
	\draw (1.4, 0) -- (1.7, 0.08);  % справа

	\begin{scriptsize}
		\draw [fill=black] (0, 0) circle (1.5pt);        % точка в центре
		\draw [fill=black] (-1, 1.45) circle (1.5pt);    % точка вверху слева
		\draw [fill=black] (1, 1.45) circle (1.5pt);     % точка вверху справа
		\draw [fill=black] (2.1, -1.45) circle (1.5pt);  % точка снизу справа
		\draw [fill=black] (-2.1, -1.45) circle (1.5pt); % точка снизу слева
		\draw [fill=black] (1, 1.05) circle (1.5pt);     % точка радиуса вписанной окружности
		\draw [fill=black] (0, 1.45) circle (1.5pt);     % точка вверху в центре
		\draw [fill=black] (0, -1.45) circle (1.5pt);    % точка внизу в центре

		\draw [color=black] (-0.2, 0) node {$O$};
		\draw (0.6, 0.4) node {$R$};
		\draw (-0.7, 1.1) node {$\alpha$};
		\draw (-1.6, -1.1) node {$\beta$};
	\end{scriptsize}
\end{tikzpicture}
\end{center}
\end{figure}

\subsection{Обозначения}

R - радиус вписанной окружности \\
D - диаметр вписанной окружности \\
O - центр вписанной окружности \\
H - высота трапеции \\
$\alpha, \beta$ - углы трапеции

\subsection{Формула площади равнобедренной трапеции через радиус вписанной окружности}

\[ S = \frac{R^2}{4}\sin\alpha = \frac{R^2}{4}\sin\beta \]

\subsection{СПРАВЕДЛИВО, для вписанной окружности в равнобедренную трапецию}

\[ H = D = 2R \] \\

\newpage
\section{Формула площади равнобедренной трапеции через диагонали и угол между ними}

\begin{figure}[h]
\begin{center}
	\begin{tikzpicture}
	\draw (-1, 1.45) -- (2.1, -1.45); % диагональ
	\draw (1, 1.45) -- (-2.1, -1.45); % диагональ
	
	%\draw [shift={(0, 0.45)}] (109 : 0.35) arc (109 : 170 : 0.35); % нижний правый угол
	
	\draw(0.01, 0.52) circle (0.25cm); % круг	
	\draw [shift={(0, 0.51)}] (0, 0) -- (136 : 0.35) arc (140 : 222 : 0.35) -- cycle; % угол слева
	\draw [shift={(0, 0.51)}] (0, 0) -- (-42 : 0.35) arc (-42 : 43 : 0.35) -- cycle;  % угол справа

	\draw (-1, 1.45) -- (1, 1.45);       % верхняя линия
	\draw (1, 1.45) -- (2.1, -1.45);     % линия справа
	\draw (2.1, -1.45) -- (-2.1, -1.45); % нижняя линия
	\draw (-2.1, -1.45) -- (-1, 1.45);   % линия слева

	\draw (-1.7, 0) -- (-1.4, -0.08); % слева
	\draw (-1.68, 0.08) -- (-1.4, 0); % слева

	\draw (1.4, -0.08) -- (1.7, 0); % справа
	\draw (1.4, 0) -- (1.7, 0.08);  % справа

	\begin{scriptsize}
		\draw [fill=black] (-1, 1.45) circle (1.5pt);    % точка вверху слева
		\draw [fill=black] (1, 1.45) circle (1.5pt);     % точка вверху справа
		\draw [fill=black] (2.1, -1.45) circle (1.5pt);  % точка снизу справа
		\draw [fill=black] (-2.1, -1.45) circle (1.5pt); % точка снизу слева
		
		\draw (-0.5, 0.5) node {$\alpha$};
		\draw (0, 0) node {$\beta$};
		\draw (-1, -0.75) node {$d$};
		\draw (1, -0.75) node {$d$};
	\end{scriptsize}
\end{tikzpicture}
\end{center}
\end{figure}

\subsection{Обозначения}

d - диагональ трапеции \\
$\alpha, \beta$ - углы между диагоналями

\subsection{Формула площади равнобедренной трапеции через диагонали и угол между ними}

\[ S = \frac{d^2}{2}\sin\alpha = \frac{d^2}{2}\sin\beta \]

\newpage
\section{Формула площади равнобедренной трапеции через среднюю линию, боковую сторону и угол при основании}

\begin{figure}[h]
\begin{center}
	\begin{tikzpicture}

	\draw [shift={(-1, 1.45)}] (0, 0) -- (-109 : 0.25) arc (-109 : 0 : 0.25) -- cycle; % левый верхний угол
	\draw [shift={(1, 1.45)}] (0, 0) -- (180 : 0.25) arc (180 : 289 : 0.25) -- cycle;  % правый верхний угол

	\draw (-1, 1.45) -- (1, 1.45);       % верхняя линия
	\draw (1, 1.45) -- (2.1, -1.45);     % линия справа
	\draw (2.1, -1.45) -- (-2.1, -1.45); % нижняя линия
	\draw (-2.1, -1.45) -- (-1, 1.45);   % линия слева
	
	\draw (-1.55, 0) -- (1.55, 0); % средняя линия трапеции

	\draw [shift={(2.1, -1.45)}] (109 : 0.35) arc (109 : 180 : 0.35); % нижний правый угол
	\draw [shift={(2.1,-1.45)}] (109 : 0.3) arc (109 : 180 : 0.3);    % нижний правый угол

	\draw [shift={(-2.1, -1.45)}] (0 : 0.35) arc (0 : 70 : 0.35);     % нижний левый угол
	\draw [shift={(-2.1, -1.45)}] (0 : 0.3) arc (0 : 70 : 0.3);       % нижний левый угол

	\draw (-1.8, -0.2) -- (-1.45, -0.3);   % слева
	\draw (-1.85, -0.35) -- (-1.5, -0.45); % слева
	\draw (1.8, -0.2) -- (1.45, -0.3);     % справа
	\draw (1.85, -0.35) -- (1.5, -0.45);   % справа

	\begin{scriptsize}
		\draw [fill=black] (-1, 1.45) circle (1.5pt);    % точка вверху слева
		\draw [fill=black] (1, 1.45) circle (1.5pt);     % точка вверху справа
		\draw [fill=black] (2.1, -1.45) circle (1.5pt);  % точка снизу справа
		\draw [fill=black] (-2.1, -1.45) circle (1.5pt); % точка снизу слева
		
		\draw [fill=black] (-1.55, 0) circle (1.5pt); % точка в центре слева
		\draw [fill=black] (1.55, 0) circle (1.5pt);  % точка в центре справа

		\draw (-0.7, 1.1) node {$\alpha$};
		\draw (-1.6, -1.1) node {$\beta$};
		\draw (-1.8, 0) node {$с$};
		\draw (1.8, 0) node {$с$};
		\draw (0, -0.2) node {$m$};
	\end{scriptsize}
\end{tikzpicture}
\end{center}
\end{figure}

\subsection{Обозначения}

c - боковая сторона \\
m - средняя линия трапеции \\
$\alpha, \beta$ - углы между диагоналями

\subsection{Формула площади равнобедренной трапеции через среднюю линию, боковую сторону и угол при основании}

\[ S = mc\sin\alpha = mc\sin\beta \]

\newpage
\section{Формула площади равнобедренной трапеции через основания и высоту}

\begin{figure}[h]
\begin{center}
	\begin{tikzpicture}

	\draw [shift={(-1, 1.45)}] (0, 0) -- (-109 : 0.25) arc (-109 : 0 : 0.25) -- cycle; % левый верхний угол
	\draw [shift={(1, 1.45)}] (0, 0) -- (180 : 0.25) arc (180 : 289 : 0.25) -- cycle;  % правый верхний угол

	\draw (-1, 1.45) -- (1, 1.45);       % верхняя линия
	\draw (1, 1.45) -- (2.1, -1.45);     % линия справа
	\draw (2.1, -1.45) -- (-2.1, -1.45); % нижняя линия
	\draw (-2.1, -1.45) -- (-1, 1.45);   % линия слева
	
	\draw (-1, 1.45) -- (-1, -1.45); % высота трапеции

	\draw [shift={(2.1, -1.45)}] (109 : 0.35) arc (109 : 180 : 0.35); % нижний правый угол
	\draw [shift={(2.1,-1.45)}] (109 : 0.3) arc (109 : 180 : 0.3);    % нижний правый угол

	\draw [shift={(-2.1, -1.45)}] (0 : 0.35) arc (0 : 70 : 0.35);     % нижний левый угол
	\draw [shift={(-2.1, -1.45)}] (0 : 0.3) arc (0 : 70 : 0.3);       % нижний левый угол

	\draw (-1.7, 0) -- (-1.4, -0.08); % слева
	\draw (-1.68, 0.08) -- (-1.4, 0); % слева
	\draw (1.4, -0.08) -- (1.7, 0);   % справа
	\draw (1.4, 0) -- (1.7, 0.08);    % справа

	\begin{scriptsize}
		\draw [fill=black] (-1, 1.45) circle (1.5pt);    % точка вверху слева
		\draw [fill=black] (1, 1.45) circle (1.5pt);     % точка вверху справа
		\draw [fill=black] (2.1, -1.45) circle (1.5pt);  % точка снизу справа
		\draw [fill=black] (-2.1, -1.45) circle (1.5pt); % точка снизу слева
		
		\draw [fill=black] (-1, -1.45) circle (1.5pt);  % нижняя точка высоты трапеции

		\draw (0, 1.6) node {$a$};
		\draw (0, -1.65) node {$b$};
		\draw (-0.75, 0) node {$h$};
	\end{scriptsize}
\end{tikzpicture}
\end{center}
\end{figure}

\subsection{Обозначения}

a - нижнее основание \\
b - верхнее основание \\
h - высота трапеции

\subsection{Формула площади равнобедренной трапеции через основания и высоту}

\[ S = \frac{a + b}{a}h \]

\newpage
\section{Площадь треугольника по стороне и двум углам}

\begin{figure}[h]
\begin{center}
	\begin{tikzpicture}
	
		\draw  (0, 0) -- (1.5, 3); % (A, B)
		\draw  (1.5, 3) -- (4, 1); % (B, C)
		\draw  (4, 1) -- (0, 0);   % (A, C)
		
		\draw [shift={(0, 0)}] (0, 0) -- (14 : 0.6) arc (14 : 63.4 : 0.6) -- cycle; % угол A
		\draw [shift={(1.5, 3)}] (-116.5 : 0.6) arc (-116.5 : -38.6 : 0.6);         % угол В
		\draw [shift={(1.5, 3)}] (-116.5 : 0.5) arc (-116.5 : -38.6 : 0.5);         % угол В
		\draw [shift={(4, 1)}] (141.3 : 0.6) arc (141.3 : 194 : 0.6);               % угол С
		\draw [shift={(4, 1)}] (141.3 : 0.5) arc (141.3 : 194 : 0.5);               % угол С
		\draw [shift={(4, 1)}] (141.3 : 0.4) arc (141.3 : 194 : 0.4);               % угол С
		
	\begin{scriptsize}
		\draw [fill=black] (0, 0) circle (1.5pt);   % точка А
		\draw [fill=black] (1.5, 3) circle (1.5pt); % точка В
		\draw [fill=black] (4, 1) circle (1.5pt);   % точка С
		
		\draw (3, 2) node {$a$};
		\draw (2, 0.25) node {$b$};
		\draw (0.5, 1.5) node {$c$};
		
		\draw (0.75, 0.5) node {$\alpha$};
		\draw (1.65, 2.15) node {$\beta$};
		\draw (3.25, 1.15) node {$\gamma$};
	\end{scriptsize}
	\end{tikzpicture}
\end{center}
\end{figure}

\subsection{Обозначения}

a, b, c- стороны треугольника \\
$\alpha, \beta, \gamma$ - противолежащие углы

\subsection{Площадь треугольника через сторону и два угла}

\begin{align*}
S = \frac{a^2}{2} \times \frac{\sin(\beta)\sin(\gamma)}{\sin(\beta + \gamma)} = \frac{a^2}{2} \times \frac{\sin(\beta)\sin(\gamma)}{\sin(\alpha)} \\[6pt]
S = \frac{b^2}{2} \times \frac{\sin(\alpha)\sin(\gamma)}{\sin(\alpha + \gamma)} = \frac{b^2}{2} \times \frac{\sin(\alpha)\sin(\gamma)}{\sin(\beta)} \\[6pt]
S = \frac{c^2}{2} \times \frac{\sin(\alpha)\sin(\beta)}{\sin(\alpha + \beta)} = \frac{c^2}{2} \times \frac{\sin(\alpha)\sin(\beta)}{\sin(\gamma)} \\
\end{align*}

\newpage
\section{Формула площади правильного многоугольника}

\begin{figure}[h]
\begin{center}
	\begin{tikzpicture}
		\draw (0, 4) -- (-1.5, 3);
		\draw (-1.5, 3) -- (-1, 1);
		\draw (-1, 1) -- (1, 1);
		\draw (1, 1) -- (1.5, 3);
		\draw (1.5, 3) -- (0, 4);
		\draw (0, 4) -- (0, 2.5);
		\draw (-1.5, 3) -- (0, 2.5);
		\draw (1.5, 3) -- (0, 2.5);
		\draw (-1, 1) -- (0, 2.5);
		\draw (1, 1) -- (0, 2.5);
	\begin{scriptsize}
		\draw [fill=black] (0, 4) circle (1.5pt);
		\draw [fill=black] (-1.5, 3) circle (1.5pt);
		\draw [fill=black] (-1, 1) circle (1.5pt);
		\draw [fill=black](1, 1) circle (1.5pt);
		\draw [fill=black] (1.5, 3) circle (1.5pt);
		\draw [fill=black] (0, 2.5) circle (1.5pt);
		
		\draw (0.8, 3.75) node {$a$};
		\draw (1.5, 2) node {$a$};
		\draw (-0.8, 3.75) node {$a$};
		\draw (-1.5, 2) node {$a$};
		\draw (0, 0.8) node {$a$};		
	\end{scriptsize}
	\end{tikzpicture}
\end{center}
\end{figure}

\subsection{Обозначения}

a - сторона многоугольника \\
n - количество сторон

\subsection{Площадь правильного многоугольника}

\[ S = \frac{na^2}{4\tg \left(\dfrac{180\,^{\circ}}{n} \right)} \]

\newpage
\section{Площадь треугольника, формула Герона}

\begin{figure}[h]
\begin{center}
	\begin{tikzpicture}
		\draw  (0, 0) -- (1.5, 3); % (A, B)
		\draw  (1.5, 3) -- (4, 1); % (B, C)
		\draw  (4, 1) -- (0, 0);   % (A, C)
	\begin{scriptsize}
		\draw [fill=black] (0, 0) circle (1.5pt);   % точка А
		\draw [fill=black] (1.5, 3) circle (1.5pt); % точка В
		\draw [fill=black] (4, 1) circle (1.5pt);   % точка С
		
		\draw (3, 2) node {$a$};
		\draw (2, 0.25) node {$b$};
		\draw (0.5, 1.5) node {$c$};
	\end{scriptsize}
	\end{tikzpicture}
\end{center}
\end{figure}

\subsection{Обозначения}

a, b, c,- стороны треугольника \\[6pt]
p - полупериметр \\[6pt]
$p = \dfrac{( a + b + c)}{2}$

\subsection{Площадь правильного многоугольника}

\[ S = \sqrt{p(p - a)(p - b)(p - c)} \]

\newpage
\section{Формула расчета площади треугольника}

\begin{figure}[h]
\begin{center}
	\begin{tikzpicture}
		\draw (0, 0) -- (3, 3);
		\draw (3, 3) -- (5, 0);
		\draw (5, 0) -- (0, 0);
		\draw (3, 3) -- (3, 0);
	\begin{scriptsize}
		\draw [fill=black] (0, 0) circle (1.5pt);
		\draw [fill=black] (3, 3) circle (1.5pt);
		\draw [fill=black] (5, 0) circle (1.5pt);
		\draw [fill=black] (3, 0) circle (1.5pt);
		
		\draw (2,-0.3) node {$a$};
		\draw (2.75,1.5) node {$h$};
	\end{scriptsize}
	\end{tikzpicture}
\end{center}
\end{figure}

\subsection{Обозначения}

h - высота треугольника \\
a - основание

\subsection{Площадь правильного многоугольника}

\[ S = \frac{1}{2}ah \]

\newpage
\section{Площадь сектора кольца}

\begin{figure}[h]
\begin{center}
	\begin{tikzpicture}
		\draw (0, 0) -- (3, 0);     % горизонтальная линия
		\draw (3, 0) -- (4.5, 2.5); % вертикальная линия
		
		\draw (1.1, 0) -- (1.1, -0.5);    % вертикальная линия внизу слева
		\draw (3, 0) -- (3.0, -0.5);      % вертикальная линия внизу справа
		\draw (1.1, -0.4) -- (1.9, -0.4); % горизантальная линия внизу слева
		\draw (2.2, -0.4) -- (3.0, -0.4); % горизантальная линия внизу справа
		
		\draw (4.28, 2.15) -- (4.78, 1.85); % диагональная линия вверху справа
		\draw (3, 0)-- (3.5, -0.3);         % диагональная линия внизу справа
		
		\draw (4.65, 1.94) -- (4.08, 1);
		\draw (3.35, -0.2) -- (3.9, 0.7);
		
		%%%%%%%%--------%%%%%%%%
		\draw (3.5, 2.45) -- (2.52, 2.45);  % 01
		\draw (3.98, 2.3) -- (2.02, 2.3);   % 02
		\draw (4.28, 2.15) -- (1.73, 2.15); % 03
		\draw (4.2, 2) -- (1.5, 2);         % 04
		\draw (4.11, 1.85) -- (1.32, 1.85); % 05
		\draw (4.03, 1.7) -- (1.18, 1.7);   % 06
		\draw (3.94, 1.55) -- (1.04, 1.55); % 07
		\draw (3.85, 1.4) -- (0.93, 1.4);   % 08
		\draw (3.76, 1.25) -- (0.83, 1.25); % 09
		\draw (3.67, 1.1) -- (0.75, 1.1);   % 10
		\draw (3.58, 0.95) -- (0.69, 0.95); % 11
		\draw (3.49, 0.8) -- (0.64, 0.8);   % 12
		\draw (3.4, 0.65) -- (0.58, 0.65);  % 13
		\draw (3.31, 0.5) -- (0.55, 0.5);   % 14
		\draw (3.22, 0.35) -- (0.53, 0.35); % 15
		\draw (3.13, 0.2) -- (0.51, 0.2);   % 16
		\draw (3.04, 0.05) -- (0.5, 0.05);  % 17
		
		\draw [shift={(3, 0)}, fill=white] (0,0) -- plot[domain=1.03 : 3.14, variable=\t]({1*1.9 * cos(\t r) +- 0 * 1.9 * sin(\t r)}, {0 * 1.9 * cos(\t r) + 1 * 1.9 * sin(\t r)}) -- cycle; % малая дуга
		\draw [shift={(3, 0)}] (0, 0) -- plot[domain=1.03 : 3.14, variable=\t]({1 * 2.5 * cos(\t r) +- 0 * 2.5 * sin(\t r)}, {0 * 2.5 * cos(\t r) + 1 * 2.5 * sin(\t r)}) -- cycle;     % большая дуга
	\begin{scriptsize}
		\draw [fill=black] (1.1, 0) circle (1.5pt);
		\draw [fill=black] (3.97, 1.63) circle (1.5pt);
		\draw [fill=black] (4.29, 2.14) circle (1.5pt);
		\draw [fill=black] (0.5, 0) circle (1.5pt); % точка A
		\draw [fill=black] (3, 0) circle (1.5pt); % точка O
		
		\draw (0.45, -0.35) node {$A$};
		\draw (2.05, -0.4) node {$r$};
		\draw (3.2, -0.35) node {$O$};		
		\draw (2.9, 0.25) node {$\alpha$};
		\draw (4, 0.85) node {$R$};
		\draw (4.2, 2.4) node {$B$};
	\end{scriptsize}
	\end{tikzpicture}
\end{center}
\end{figure}

\subsection{Обозначения}

R - радиус внешней окружности \\
r - радиус внутренней окружности \\
$\alpha$ - угол сектора AOB, в градусах

\subsection{Формула площади сектора кольца}

\[ S = \frac{\pi\alpha}{360\,^{\circ}}(R^2 - r^2) \]

\newpage
\section{Площадь кольца}

\begin{figure}[h]
\begin{center}
	\begin{tikzpicture}
		\draw(0, 0) circle (1cm);
		\draw(0, 0) circle (1.8cm);
		
		\draw (0, 0) -- (1.5, 1);
		\draw (0, 0) -- (-0.87, 0.5);
	\begin{scriptsize}
		\draw [fill=black] (0, 0) circle (1.5pt); % точка O
		\draw [fill=black] (-0.87, 0.5) circle (1.5pt); % точка r
		\draw [fill=black] (1.5, 1) circle (1.5pt); % точка R
		
		\draw (0, -0.2) node {$O$};
		\draw (-0.5, 0.1) node {$r$};
		\draw (1.25, 0.55) node {$R$};
	\end{scriptsize}
	\end{tikzpicture}
\end{center}
\end{figure}

\subsection{Обозначения}

R - радиус внешней окружности \\
r - радиус внутренней окружности

\subsection{Формула площади кольца}

\[ S = \pi(R^2 - r^2) \]

\newpage
\section{Площадь сегмента круга}

\begin{figure}[h]
\begin{center}
	\begin{tikzpicture}
		\draw [shift={(0, 0)}] (0, 0) -- (80 : 0.4) arc (80 : 180 : 0.4) -- cycle;
		
		\draw (0, 0) circle (2cm);
		
		\draw (0, 0) -- (0.35, 1.97);
		\draw (0.35, 1.97) -- (-2, 0);
		\draw (-2, 0) -- (0, 0);
		
		\draw (-0.45, 1.95) -- (0.4, 1.95);   % 01
		\draw (-0.87, 1.8) -- (0.15, 1.8);    % 02
		\draw (-1.13, 1.65) -- (-0.03, 1.65); % 03
		\draw (-1.32, 1.5) -- (-0.21, 1.5);   % 04
		\draw (-1.48, 1.35) -- (-0.39, 1.35); % 05
		\draw (-1.6, 1.20) -- (-0.57, 1.2);   % 06
		\draw (-1.7, 1.05) -- (-0.75, 1.05);  % 07
		\draw (-1.78, 0.9) -- (-0.93, 0.9);   % 08
		\draw (-1.85, 0.75) -- (-1.11, 0.75); % 09
		\draw (-1.91, 0.6) -- (-1.29, 0.6);   % 10
		\draw (-1.95, 0.45) -- (-1.47, 0.45); % 11
		\draw (-1.98, 0.3) -- (-1.64, 0.3);   % 12
		\draw (-2, 0.15) -- (-1.82, 0.15);    % 13	
	\begin{scriptsize}
		\draw [fill=black] (0, 0) circle (1.5pt);
		\draw [fill=black] (0.34, 1.96) circle (1.5pt);
		\draw [fill=black] (-2, 0) circle (1.5pt);
		
		\draw (0, -0.2) node {$O$};
		\draw (0.4, 2.2) node {$C$};
		\draw (-2.2, 0) node {$A$};
		\draw (-0.35, 0.4) node {$\alpha$};
		\draw (-1, -0.2) node {$R$};
		\draw (0.34, 0.85) node {$R$};
	\end{scriptsize}
	\end{tikzpicture}
\end{center}
\end{figure}

\subsection{Обозначения}

R - радиус круга \\
$\alpha$ - угол сегмента в градусах

\subsection{Формула площади сегмента круга, отсекаемая хордой AC}

\[ S = \frac{1}{2}R^2 \left( \frac{\pi\alpha}{180\,^{\circ}} - \sin\alpha \right) \]

\newpage
\section{Площадь сектора круга}

\begin{figure}[h]
\begin{center}
	\begin{tikzpicture}
		\draw (0, 0) -- (3, 0);     % горизонтальная линия
		\draw (3, 0) -- (4.5, 2.5); % вертикальная линия
		
		\draw [shift={(3, 0)}] (0, 0) -- plot[domain=1.03 : 3.14, variable=\t]({1 * 2.5 * cos(\t r) +- 0 * 2.5 * sin(\t r)}, {0 * 2.5 * cos(\t r) + 1 * 2.5 * sin(\t r)}) -- cycle; % большая дуга
		
		\draw [shift={(3, 0)}] (0, 0) -- (59 : 0.5) arc (58 : 180.2 : 0.5) -- cycle;
	\begin{scriptsize} % угол
	
		\draw [fill=black] (4.29, 2.14) circle (1.5pt); % точка B
		\draw [fill=black] (0.5, 0) circle (1.5pt);     % точка A
		\draw [fill=black] (3, 0) circle (1.5pt);       % точка O
		
		\draw (0.45, -0.35) node {$A$};
		\draw (3.2, -0.35) node {$O$};		
		\draw (4.2, 2.4) node {$B$};
		
		\draw (1.15, 2) node {$L$};
		\draw (3.8, 0.85) node {$r$};
		\draw (2.6, 0.6) node {$\alpha$};
	\end{scriptsize}
	\end{tikzpicture}
\end{center}
\end{figure}

\subsection{Обозначения}

r - радиус круга \\
L - длина дуги AB \\
$\alpha$ - угол сектора круга AOB в градусах

\subsection{Формула площади сектора круга, через длину дуги (L)}

\[ S = \frac{1}{2}Lr \]

\subsection{Формула площади сектора круга, через угол ($\alpha$):}

\[ S = \frac{\pi r^2 \alpha}{360\,^{\circ}} \]

\newpage
\section{Площадь ромба}

\begin{figure}[h]
\begin{center}
	\begin{tikzpicture}
		\draw  (0, 0) -- (1, 2);  % верхняя левая сторона ромба
		\draw  (1, 2)-- (2, 0);   % верхняя правая сторона ромба
		\draw  (2,0) -- (1,-2);   % нижняя правая сторона ромба
		\draw  (1, -2) -- (0, 0); % нижняя левая сторона ромба
		
		\draw (0, 0) -- (2,0);   % горизонтальная линия
		\draw (1, 2) -- (1, -2); % вертикальная линия
		
		\draw [shift={(0, 0)}] (0, 0) -- (-63 : 0.4) arc (-63 : 63 : 0.4) -- cycle; % правый угол
		\draw [shift={(2,0)}] (0, 0) -- (116 : 0.4) arc (116 : 243 : 0.4) -- cycle; % левый угол
		\draw (1, 0) -- (1.2, 0) -- (1.2, 0.2) -- (1, 0.2) -- cycle;                % угол 90
		
		\draw [shift={(1,2)}] (-116 : 0.6) arc (-116 : -63 : 0.6); % верхний угол
		\draw [shift={(1,2)}] (-116 : 0.5) arc (-116 : -63 : 0.5); % верхний угол
		\draw [shift={(1, -2)}] (63 : 0.6) arc (63 : 116 : 0.6);   % нижний угол
		\draw [shift={(1, -2)}] (63 : 0.5) arc (63 : 116 : 0.5);   % нижний угол
	\begin{scriptsize}
		\draw [fill=black] (0, 0) circle (1.5pt);  % левая точка
		\draw [fill=black] (1, 2) circle (1.5pt);  % верхняя точка
		\draw [fill=black] (2, 0) circle (1.5pt);  % правая точка
		\draw [fill=black] (1, -2) circle (1.5pt); % нижняя точка
		\draw [fill=black] (1, 0) circle (1.5pt);  % точка в центре
		
		\draw (0.2, 1.15) node {$a$};  % левая верхняя a
		\draw (1.8, 1.15) node {$a$};  % правая верхняя a
		\draw (1.8, -1.15) node {$a$}; % правая нижняя a
		\draw (0.2, -1.15) node {$a$}; % левая нижняя a
		
		\draw (1.15, -0.6) node {$D$};
		\draw (0.6, 0.2) node {$d$};
		
		\draw (1.5, -0.2) node {$\alpha$};
		\draw (1.15, 1.2) node {$\beta$};
	\end{scriptsize}
	\end{tikzpicture}
\end{center}
\end{figure}

\subsection{Обозначения}

a - сторона ромба \\
D - большая диагональ \\
d - меньшая диагональ \\
$\alpha$ - острый угол \\
$\beta$ - тупой угол

\subsection{Формулы площади ромба}

\begin{align*}
S = \frac{D \times d}{2} \\[6pt]
S = a^2\sin\alpha = a^2\sin\beta \\[6pt]
S = \frac{1}{2} D^2 \tg \left( \frac{\alpha}{2} \right) = \frac{1}{2} d^2 \tg \left( \frac{\beta}{2} \right)
\end{align*}

\newpage
\section{Формула площади трапеции через четыре стороны}

\begin{figure}[h]
\begin{center}
	\begin{tikzpicture}
		\draw  (0, 0) -- (1.5, 2.5);   % сторона a
		\draw  (1.5, 2.5) -- (4, 2.5); % сторона b
		\draw  (4, 2.5) -- (4.5, 0);   % сторона c
		\draw  (4.5, 0) -- (0, 0);     % сторона d
	\begin{scriptsize}
		\draw [fill=black] (0, 0) circle (1.5pt);     % нижняя левая точка
		\draw [fill=black] (1.5, 2.5) circle (1.5pt); % верхняя левая точка
		\draw [fill=black] (4, 2.5) circle (1.5pt);   % верхняя правая точка
		\draw [fill=black] (4.5, 0) circle (1.5pt);   % нижняя правая точка

		\draw (0.55, 1.3) node {$a$};
		\draw (2.7, 2.8) node {$b$};
		\draw (4.45, 1.3) node {$c$};
		\draw (2.25, -0.3) node {$d$};
	\end{scriptsize}
	\end{tikzpicture}
\end{center}
\end{figure}

\subsection{Обозначения}

d - нижнее основание \\
b - верхнее основание \\
a, d - боковые стороны

\subsection{Формулы площади трапеции}

\[ S = \frac{a + b}{2} \sqrt{c^2 - \left( \frac{(a - b)^2 + c^2 -d^2}{2(a - b)} \right)^2} \]

\newpage
\section{Формула площади параллелограмма через стороны и углы}

\begin{figure}[h]
\begin{center}
	\begin{tikzpicture}
		\draw  (0, 0) -- (1, 2); % сторона a влево
		\draw  (1, 2) -- (4, 2); % сторона в вверху
		\draw  (4, 2) -- (3, 0); % сторона a вправо
		\draw  (3, 0) -- (0, 0); % сторона в внизу
		
		\draw [shift={(3, 0)}] (0, 0) -- (63 : 0.5) arc (63 : 180 : 0.5) -- cycle;
		\draw [shift={(1, 2)}] (0, 0) -- (-116 : 0.5) arc (-116 : 0 : 0.5) -- cycle;
		
		\draw [shift={(4, 2)}] (180 : 0.5) arc (180 : 243 : 0.5);
		\draw [shift={(4, 2)}] (180 : 0.6) arc (180 : 243 : 0.6);
		\draw [shift={(0, 0)}] (0 : 0.5) arc (0 : 63 : 0.5);
		\draw [shift={(0, 0)}] (0 : 0.6) arc (0 : 63 : 0.6);
	\begin{scriptsize}
		\draw [fill=black] (0, 0) circle (1.5pt);
		\draw [fill=black] (1, 2) circle (1.5pt);
		\draw [fill=black] (4, 2) circle (1.5pt);
		\draw [fill=black] (3, 0) circle (1.5pt);
		
		\draw (0.25, 1) node {$a$};
		\draw (3.75, 1) node {$a$};
		\draw (2.5, 2.2) node {$b$};
		\draw (1.5, -0.2) node {$b$};
		
		\draw (2.45, 0.5) node {$\alpha$};
		\draw (0.75, 0.5) node {$\beta$};		
	\end{scriptsize}
	\end{tikzpicture}
\end{center}
\end{figure}

\subsection{Обозначения}

a, b - стороны параллелограмма \\
$\alpha, \beta$ - углы параллелограмма

\subsection{Формула площади через стороны и углы параллелограмма}

\[ S = ab \times \sin\alpha = ab \times \sin\beta \]

\newpage
\section{Формула площади параллелограмма через сторону и высоту}

\begin{figure}[h]
\begin{center}
	\begin{tikzpicture}
		\draw  (0, 0) -- (1, 2); % левая сторона a
		\draw  (1, 2) -- (4, 2); % верхняя сторона в
		\draw  (4, 2) -- (3, 0); % правая сторона a
		\draw  (3, 0) -- (0, 0); % нижняя сторона в
		
		\draw (1, 2) -- (1, 0);     % вертикальная линия
		\draw (1, 2) -- (3.4, 0.8); % диагональная линия
		
		\draw (1, 0) -- (1.25, 0) -- (1.25, 0.25) -- (1, 0.25) -- cycle;             % угол 90 у вертикальной линии
		\draw (3.155, 0.92) -- (3.4, 0.8) -- (3.525, 1.05) -- (3.28, 1.17) -- cycle; % угол 90 у горизонтальной линии
	\begin{scriptsize}
		\draw [fill=black] (0, 0) circle (1.5pt);
		\draw [fill=black] (1, 2) circle (1.5pt);
		\draw [fill=black] (4, 2) circle (1.5pt);
		\draw [fill=black] (3, 0) circle (1.5pt);
		\draw [fill=black] (1, 0) circle (1.5pt);
		\draw [fill=black] (3.4, 0.8) circle (1.5pt);

		\draw (0.25, 1) node {$a$};
		\draw (3.75, 1) node {$a$};
		\draw (2.5, 2.2) node {$b$};
		\draw (1.5, -0.2) node {$b$};
		
		\draw (1.25, 0.85) node {$H_b$};
		\draw (2.1, 1.25) node {$H_a$};
	\end{scriptsize}
	\end{tikzpicture}
\end{center}
\end{figure}

\subsection{Обозначения}

a, b - стороны параллелограмма \\
$H_b$ - высота на сторону b \\
$H_a$ - высота на сторону a

\subsection{Формула площади через стороны и высоты параллелограмма}

\[ S = b \times H_b = a \times H_a \]

\newpage
\section{Формула площади параллелограмма через диагонали и угол между ними}

\begin{figure}[h]
\begin{center}
	\begin{tikzpicture}
		\draw  (0, 0) -- (1, 2);
		\draw  (1, 2) -- (4, 2);
		\draw  (4, 2) -- (3, 0);
		\draw  (3, 0) -- (0, 0);
		
		\draw (1, 2) -- (3, 0);
		\draw (0, 0) -- (4, 2);
		
		\draw [shift={(2, 1)}] (0, 0) -- (135 : 0.5) arc (135 : 206 : 0.5) -- cycle;
		\draw [shift={(2, 1)}] (0, 0) -- (-45 : 0.5) arc (-45 : 26 : 0.5) -- cycle;
		
		\draw(2, 1) circle (0.35cm);
	\begin{scriptsize}
		\draw [fill=black] (0, 0) circle (1.5pt);
		\draw [fill=black] (1, 2) circle (1.5pt);
		\draw [fill=black] (4, 2) circle (1.5pt);
		\draw [fill=black] (3, 0) circle (1.5pt);
		\draw [fill=black] (2, 1) circle (1.5pt);
		
		\draw (2.65, 0.9) node {$d$};
		\draw (2.65, 1.6) node {$D$};
		
		\draw (1.3, 1.1) node {$\alpha$};
		\draw (1.9, 0.45) node {$\beta$};
	\end{scriptsize}
	\end{tikzpicture}
\end{center}
\end{figure}

\subsection{Обозначения}

D - большая диагональ \\
d - меньшая диагональ \\
$\alpha, \beta$ - углы между диагоналями

\subsection{Формула площади через диагонали параллелограмма и угол между ними}

\[ S = \frac{1}{2}Dd \times \sin\alpha = \frac{1}{2}Dd \times \sin\beta \]

\newpage
\section{Найти площадь треугольника, угол и две стороны}

\begin{figure}[h]
\begin{center}
	\begin{tikzpicture}
		\draw  (0, 0) -- (1.5, 3);
		\draw  (1.5, 3) -- (4, 1);
		\draw  (4, 1) -- (0, 0); 

		\draw [shift={(4, 1)}] (0, 0) -- (141.5 : 0.3) arc (152 : 190 : 0.4) -- cycle; % угол alpha

		\draw [shift={(0, 0)}] (14 : 0.5) arc (14 : 64 : 0.5); % угол beta
		\draw [shift={(0, 0)}] (14 : 0.4) arc (14 : 64 : 0.4); % угол beta

		\draw [shift={(1.5, 3)}] (-116 : 0.6) arc (-120 :-40 : 0.6); % угол gamma
		\draw [shift={(1.5, 3)}] (-116 : 0.5) arc (-120 :-40 : 0.5); % угол gamma
		\draw [shift={(1.5, 3)}] (-116 : 0.4) arc (-120 :-40 : 0.4); % угол gamma
	\begin{scriptsize}
		\draw [fill=black] (0, 0) circle (1.5pt);
		\draw [fill=black] (1.5, 3) circle (1.5pt);
		\draw [fill=black] (4, 1) circle (1.5pt);
		
		\draw (3, 2) node {$a$};
		\draw (2, 0.25) node {$b$};
		\draw (0.5, 1.5) node {$c$};
		
		\draw (3.5, 1.1) node {$\alpha$};
		\draw (0.6, 0.5) node {$\beta$};
		\draw (1.65, 2.2) node {$\gamma$};
	\end{scriptsize}
	\end{tikzpicture}
	\caption{Зная у треугольника, две стороны и синус угла между ними, находим по формуле, его площадь.}
\end{center}
\end{figure}

\subsection{Обозначения}

a, b, c - стороны треугольника \\
$\alpha, \beta, \gamma$ - углы

\subsection{Формулы площади треугольника, через две стороны и угол между ними}

\begin{align*}
S = \frac{1}{2}bc\sin(\alpha) \\[6pt]
S = \frac{1}{2}ac\sin(\beta) \\[6pt]
S = \frac{1}{2}ab\sin(\gamma)
\end{align*}

\newpage
\section{Площадь равностороннего треугольника}

\begin{figure}[h]
\begin{center}
	\begin{tikzpicture}
		\draw (0, 0) -- (2, 3);
		\draw (2, 3) -- (4, 0);
		\draw (4, 0) -- (0, 0);
		\draw (2, 3) -- (2, 0);
		
		\draw (2, 0) -- (2.25, 0) -- (2.25, 0.25) -- (2, 0.25) -- cycle; 
	\begin{scriptsize}
		\draw [fill=black] (0, 0) circle (1.5pt);
		\draw [fill=black] (2, 3) circle (1.5pt);
		\draw [fill=black] (4, 0) circle (1.5pt);
		\draw [fill=black] (2, 0) circle (1.5pt);
		
		\draw (0.65, 1.5) node {$a$};
		\draw (3.3, 1.5) node {$a$};
		\draw (1.4, -0.2) node {$a$};		
		\draw (2.2, 1.3) node {$h$};
	\end{scriptsize}
	\end{tikzpicture}
\end{center}
\end{figure}

\subsection{Обозначения}
a - сторона треугольника \\
h - высота

\subsection{Площадь треугольника через сторону a и высоту h}

\[ S = \frac{1}{2}ah \]

\subsection{Площадь треугольника только через сторону a}

\[ S = \frac{\sqrt{3}}{4}a^2 \]

\subsection{Площадь треугольника только через высоту h}

\[ S = \frac{h^2}{\sqrt{3}} \]

\newpage
\section{площадь равнобедренного треугольника}

\begin{figure}[h]
\begin{center}
	\begin{tikzpicture}
		\draw  (0, 0)  -- (1.5, 3);
		\draw  (1.5, 3) -- (3, 0);
		\draw  (3, 0) -- (0, 0);
		\draw (1.5, 3) -- (1.5, 0);
		
		\draw (1.5, 0) -- (1.75, 0) -- (1.75, 0.25) -- (1.5, 0.25);
	\begin{scriptsize}
		\draw [fill=black] (0, 0) circle (1.5pt);
		\draw [fill=black] (1.5,3.0) circle (1.5pt);
		\draw [fill=black] (3.0,0.0) circle (1.5pt);
		\draw [fill=black] (1.5, 0) circle (1.5pt);
		
		\draw (0.5, 1.5) node {$a$};
		\draw (2.5, 1.5) node {$a$};
		\draw (1, -0.2) node {$b$};		
		\draw (1.7, 1.3) node {$h$};
	\end{scriptsize}
	\end{tikzpicture}
\end{center}
\end{figure}

\subsection{Обозначения}

b - основание треугольника \\
a - равные стороны \\
h - высота

\subsection{Формула площади треугольника через высоту h и основание b}

\[ S = \frac{1}{2}bh \]

\subsection{Формула площади треугольника через, стороны a, b}

\[ S = \frac{b}{4}sqrt{4a^2 - b^2} \]

\newpage
\section{Формула площади трапеции через основания и высоту}

\begin{figure}[h]
\begin{center}
	\begin{tikzpicture}
		\draw  (0, 0) -- (1.5, 2.5);   % сторона a
		\draw  (1.5, 2.5) -- (4, 2.5); % сторона b
		\draw  (4, 2.5) -- (4.5, 0);   % сторона c
		\draw  (4.5, 0) -- (0, 0);     % сторона d
		
		\draw (1.5, 2.5) -- (1.5, 0);       % высота
		\draw (0.75, 1.25) -- (4.25, 1.25); % средняя линия
		
		\draw (1.5, 0) -- (1.75, 0) -- (1.75, 0.25) -- (1.5, 0.25);
	\begin{scriptsize}
		\draw [fill=black] (0, 0) circle (1.5pt);     % нижняя левая точка
		\draw [fill=black] (1.5, 2.5) circle (1.5pt); % верхняя левая точка
		\draw [fill=black] (4, 2.5) circle (1.5pt);   % верхняя правая точка
		\draw [fill=black] (4.5, 0) circle (1.5pt);   % нижняя правая точка
		
		\draw [fill=black] (1.5, 0) circle (1.5pt); 
		
		\draw [fill=black] (0.75, 1.25) circle (1.5pt); % левая точка средней линии  
		\draw [fill=black] (4.25, 1.25) circle (1.5pt); % правая точка средней линии
		
		\draw (2.55, 1.4) node {$m$};
		\draw (2.7, 2.8) node {$b$};
		\draw (1.7, 0.75) node {$h$};
		\draw (2.25, -0.3) node {$a$};
	\end{scriptsize}
	\end{tikzpicture}
\end{center}
\end{figure}

\subsection{Обозначения}

a - нижнее основание \\
b - верхнее основание \\
m - средняя линия \\
h - высота трапеции

\subsection{Формула площади трапеции}

\[ S = \frac{(a + b)}{2} \times h = m \times h \]

\newpage
\section{Формула площади трапеции через диагонали и угол между ними}

\begin{figure}[h]
\begin{center}
	\begin{tikzpicture}
		\draw (0, 0) -- (1.5, 2.5);
		\draw (1.5, 2.5) -- (4, 2.5);
		\draw (4, 2.5) -- (4.5, 0);
		\draw (4.5, 0) -- (0, 0);
		\draw (1.5, 2.5) -- (4.5, 0);
		\draw (0, 0) -- (4, 2.5);
		
		\draw [shift={(2.57, 1.6)}] (0, 0) -- (-39 : 0.5) arc (-39 : 33 : 0.5) -- cycle;
		\draw [shift={(2.57, 1.6)}] (0, 0) -- (139.5 : 0.5) arc (139.5 : 211 : 0.5) -- cycle;
		
		\draw(2.57, 1.6) circle (0.4cm);
	\begin{scriptsize}
		\draw [fill=black] (0, 0) circle (1.5pt);
		\draw [fill=black] (1.5, 2.5) circle (1.5pt);
		\draw [fill=black] (4, 2.5) circle (1.5pt);
		\draw [fill=black] (4.5, 0) circle (1.5pt);
		\draw [fill=black] (2.57, 1.6) circle (1.5pt);
		
		\draw (1.9, 1.65) node {$\alpha$};
		\draw (2.55, 1) node {$\beta$};
		\draw (1.5, 0.75) node {$d_1$};
		\draw (3.3, 0.75) node {$d_2$};
	\end{scriptsize}
	\end{tikzpicture}
\end{center}
\end{figure}
	
\subsection{Обозначения}

$d_1, d_2$ - диагонали трапеции \\
$\alpha, \beta$ - углы между диагоналями

\subsection{Формула площади трапеции}

\[ S = \frac{d_1 d_2}{2}\sin\alpha = \frac{d_1 d_2}{2}\sin\beta \]

\end{document} % Конец текста.