\documentclass[a4paper,12pt]{article}

\usepackage{cmap}					% поиск в PDF
\usepackage{mathtext} 				% русские буквы в формулах
\usepackage[T2A]{fontenc}			% кодировка
\usepackage[utf8]{inputenc}			% кодировка исходного текста
\usepackage[english,russian]{babel}	% локализация и переносы

% Дополнительная работа с математикой
\usepackage{amsmath,amsfonts,amssymb,amsthm,mathtools} % AMS
\usepackage{icomma}                                    % "Умная" запятая

%% Шрифты
\usepackage{euscript} % Шрифт Евклид
\usepackage{mathrsfs} % Матшрифт

%% Перенос знаков в формулах (по Львовскому)
\newcommand*{\hm}[1]{#1\nobreak\discretionary{}
{\hbox{$\mathsurround=0pt #1$}}{}}

%%% Работа с картинками
\usepackage{graphicx}              % Для вставки рисунков
\usepackage{wrapfig}               % Обтекание рисунков текстом

%%% Работа с таблицами
\usepackage{array,tabularx,tabulary,booktabs} % Дополнительная работа с таблицами
\usepackage{longtable}                        % Длинные таблицы
\usepackage{multirow}                         % Слияние строк в таблице

\usepackage{pgf,tikz} % Работа с графикой
\usepackage{pgfplots}
\usepackage{pgfplotstable}

\usepackage[]{geometry}

\usetikzlibrary{arrows}
\newcommand{\degre}{\ensuremath{^\circ}}

%%% Заголовок
\author{Hun Fauser}
\title{Геометрия}
\date{\today}

\begin{document} % Конец преамбулы, начало текста.

\maketitle

\newpage
\section{Площадь круга}

\begin{figure}[h]
\begin{center}
	\begin{tikzpicture}
		\draw (0, 0) circle (2.5);   % Нарисовать круг
		\draw (0, -2.5) -- (0, 2.5); % Нарисовать вертикальную линию
		\draw (0, 0) -- (2.5, 0);    % Нарисовать горизонтальную линию
	\begin{scriptsize}
		\draw (-0.25, -0.25) node {$D$};
		\draw (1, -0.25) node {$r$};
	\end{scriptsize}
	\end{tikzpicture}
	\caption{Зная диаметр или радиус круга, можно найти его площадь.}
\end{center}
\end{figure}

\subsection{Обозначения}

r = радиус круга \\
R = диаметр \\
$\pi \approx 3.14$

\subsection{Площадь круга}

\[ S = \pi r^2 = \frac{\pi}{4}D^2 \]

\subsection{Периметр круга}

\[ p = 2\pi r = \pi D \]

\newpage
\section{Площадь прямоугольника}

\begin{figure}[h]
	\begin{center}
\begin{tikzpicture}[auto]
	\draw (0, 0) -- (0, 3); %(A, B)
	\draw (0, 0) -- (5, 0); %(B, C)
	\draw (5, 0) -- (5, 3); %(C, D)
	\draw (5, 3) -- (0, 3); %(A, D)
\begin{scriptsize}
\draw (-0.25, 3.25) node {$A$};
\draw (-0.25, -0.25) node {$B$};
\draw (5.25, -0.25) node {$C$};
\draw (5.25, 3.25) node {$D$};

\draw (-0.25, 1.5) node {$a$};
\draw (2.5, -0.25) node {$b$};
\draw (5.25, 1.5) node {$c$};
\draw (2.5, 3.25) node {$d$};
\end{scriptsize}
\end{tikzpicture}
\end{center}
\end{figure}

\subsection{Обозначения}

b, d = длина прямоугольника \\
a, c = ширина прямоугольника

\subsection{Площадь прямоугольника}

\[S = ab \]

\newpage
\section{Площадь эллипса}

\begin{figure}[h]
\begin{center}
	\begin{tikzpicture}
		\draw (0, 0) ellipse (3cm and 2cm); % Нарисовать эллипс
		\draw (0, 0) -- (0, 2);     	   % Нарисовать вертикальную линию
		\draw (0, 0) -- (3, 0);    		   % Нарисовать горизонтальную линию
		\draw[dashed] (-3, 0) -- (0, 0);   % Нарисовать горизонтальную пунктирную линию
		\draw[dashed] (0, 0) -- (0, -2);   % Нарисовать вертикальную пунктирную линию
	\begin{scriptsize}
		\draw (1.5, -0.25) node {$R$};
		\draw (-0.25, 1) node {$r$};
	\end{scriptsize}
	\end{tikzpicture}
\end{center}
\end{figure}

\subsection{Обозначения}

R = большая полуось \\
r = малая полуось

\subsection{Площадь прямоугольника}

\[ S = \pi Rr \]

\newpage
\section{Формула площади равнобедренной трапеции через стороны и угол}

\begin{figure}[h]
\begin{center}
	\begin{tikzpicture}
		\draw (0, 0) -- (5, 0);     %(B, C)
		\draw (0, 0) -- (1, 2.5);   %(B, A)
		\draw (1, 2.5) -- (4, 2.5); %(A, D)
		\draw (4, 2.5) -- (5, 0);   %(D, C)
	\begin{scriptsize}		
		\draw[fill=black] (0, 0) circle (1.5pt);
		\draw[fill=black] (5, 0) circle (1.5pt);
		\draw[fill=black] (1, 2.5) circle (1.5pt);
		\draw[fill=black] (4, 2.5) circle (1.5pt);
		
		\draw (2.5, -0.25) node {$a$};
		\draw (2.5, 2.75) node {$b$};
		\draw (0.25, 1.25) node {$c$};
		\draw (4.75, 1.25) node {$c$};
		\draw (0.7, 0.5) node {$\alpha$};
		
		\draw (0.4, 1.25) -- (0.6, 1.15);  % слева
		\draw (0.35, 1.20) -- (0.55, 1.1); % слева
		\draw (4.6, 1.25) -- (4.4, 1.15);  % справа
		\draw (4.65, 1.2) -- (4.45, 1.1);  % справа
		
		\draw [shift={(1, 2.5)}] (0, 0) -- (-111.5 : 0.5) arc (-111.5 : 0 : 0.5) -- cycle; % Угол А
		\draw [shift={(4, 2.5)}] (0, 0) -- (180 : 0.5) arc (185 : 290 : 0.5) -- cycle;     % Угол D
		
		\draw [shift={(5,0)}] (111 : 0.65) arc (112 : 180 : 0.65); % Угол C
		\draw [shift={(5,0)}] (112 : 0.55) arc (112 : 180 : 0.55); % Угол C
		\draw [shift={(0,0)}] (0 : 0.65) arc (0 : 68 : 0.65);      % Угол D
		\draw [shift={(0,0)}] (0 : 0.55) arc (0 : 68 : 0.55);      % Угол D
	\end{scriptsize}
	\end{tikzpicture}
\end{center}
\end{figure}

\subsection{Обозначения}

а - нижнее основание \\
b - верхнее основание \\
с - равные боковые стороны \\
$\alpha$ - угол при нижнем основании

\subsection{Формула площади равнобедренной трапеции через стороны}

\[ S = \frac{a + b}{2}\sqrt{c^2 - \frac{(a - b)^2}{4}} \]

\subsection{Формула площади равнобедренной трапеции через стороны и угол}

\begin{align*}
S = \frac{a^2 - b^2}{4}\tg\alpha \\
S = c \times \sin\alpha(a - \cos\alpha) \\
S = c \times \sin\alpha(b + \cos\alpha)
\end{align*}

\newpage
\section{Формула площади равнобедренной трапеции через стороны и угол}

\begin{figure}[h]
\begin{center}
	\begin{tikzpicture}

	\draw [shift={(-1, 1.45)}] (0, 0) -- (-109 : 0.25) arc (-109 : 0 : 0.25) -- cycle; % левый верхний угол
	\draw [shift={(1, 1.45)}] (0, 0) -- (180 : 0.25) arc (180 : 289 : 0.25) -- cycle;  % правый верхний угол

	\draw (0.25, -1.45) -- (0.25, -1.25) -- (0, -1.25) -- (0, -1.45) -- cycle; % угол 90deg

	\draw(0, 0) circle (1.45cm); % круг

	\draw (-1, 1.45) -- (1, 1.45);       % верхняя линия
	\draw (1, 1.45) -- (2.1, -1.45);     % линия справа
	\draw (2.1, -1.45) -- (-2.1, -1.45); % нижняя линия
	\draw (-2.1, -1.45) -- (-1, 1.45);   % линия слева

	\draw [shift={(2.1, -1.45)}] (109 : 0.35) arc (109 : 180 : 0.35); % нижний правый угол
	\draw [shift={(2.1,-1.45)}] (109 : 0.3) arc (109 : 180 : 0.3);    % нижний правый угол

	\draw [shift={(-2.1, -1.45)}] (0 : 0.35) arc (0 : 70 : 0.35);     % нижний левый угол
	\draw [shift={(-2.1, -1.45)}] (0 : 0.3) arc (0 : 70 : 0.3);       % нижний левый угол

	\draw (0, 1.45) -- (0, -1.45); % диаметр вписанной окружности
	\draw (0, 0) -- (1, 1.05);     % радиус вписанной окружности

	\draw (-1.7, 0) -- (-1.4, -0.08); % слева
	\draw (-1.68, 0.08) -- (-1.4, 0); % слева

	\draw (1.4, -0.08) -- (1.7, 0); % справа
	\draw (1.4, 0) -- (1.7, 0.08);  % справа

	\begin{scriptsize}
		\draw [fill=black] (0, 0) circle (1.5pt);        % точка в центре
		\draw [fill=black] (-1, 1.45) circle (1.5pt);    % точка вверху слева
		\draw [fill=black] (1, 1.45) circle (1.5pt);     % точка вверху справа
		\draw [fill=black] (2.1, -1.45) circle (1.5pt);  % точка снизу справа
		\draw [fill=black] (-2.1, -1.45) circle (1.5pt); % точка снизу слева
		\draw [fill=black] (1, 1.05) circle (1.5pt);     % точка радиуса вписанной окружности
		\draw [fill=black] (0, 1.45) circle (1.5pt);     % точка вверху в центре
		\draw [fill=black] (0, -1.45) circle (1.5pt);    % точка внизу в центре

		\draw [color=black] (-0.2, 0) node {$O$};
		\draw (0.6, 0.4) node {$R$};
		\draw (-0.7, 1.1) node {$\alpha$};
		\draw (-1.6, -1.1) node {$\beta$};
	\end{scriptsize}
\end{tikzpicture}
\end{center}
\end{figure}

\subsection{Обозначения}

R - радиус вписанной окружности \\
D - диаметр вписанной окружности \\
O - центр вписанной окружности \\
H - высота трапеции \\
$\alpha, \beta$ - углы трапеции

\subsection{Формула площади равнобедренной трапеции через радиус вписанной окружности}

\[ S = \frac{R^2}{4}\sin\alpha = \frac{R^2}{4}\sin\beta \]

\subsection{СПРАВЕДЛИВО, для вписанной окружности в равнобедренную трапецию}

\[ H = D = 2R \] \\

\newpage
\section{Формула площади равнобедренной трапеции через диагонали и угол между ними}

\begin{figure}[h]
\begin{center}
	\begin{tikzpicture}
	\draw (-1, 1.45) -- (2.1, -1.45); % диагональ
	\draw (1, 1.45) -- (-2.1, -1.45); % диагональ
	
	%\draw [shift={(0, 0.45)}] (109 : 0.35) arc (109 : 170 : 0.35); % нижний правый угол
	
	\draw(0.01, 0.52) circle (0.25cm); % круг	
	\draw [shift={(0, 0.51)}] (0, 0) -- (136 : 0.35) arc (140 : 222 : 0.35) -- cycle; % угол слева
	\draw [shift={(0, 0.51)}] (0, 0) -- (-42 : 0.35) arc (-42 : 43 : 0.35) -- cycle;  % угол справа

	\draw (-1, 1.45) -- (1, 1.45);       % верхняя линия
	\draw (1, 1.45) -- (2.1, -1.45);     % линия справа
	\draw (2.1, -1.45) -- (-2.1, -1.45); % нижняя линия
	\draw (-2.1, -1.45) -- (-1, 1.45);   % линия слева

	\draw (-1.7, 0) -- (-1.4, -0.08); % слева
	\draw (-1.68, 0.08) -- (-1.4, 0); % слева

	\draw (1.4, -0.08) -- (1.7, 0); % справа
	\draw (1.4, 0) -- (1.7, 0.08);  % справа

	\begin{scriptsize}
		\draw [fill=black] (-1, 1.45) circle (1.5pt);    % точка вверху слева
		\draw [fill=black] (1, 1.45) circle (1.5pt);     % точка вверху справа
		\draw [fill=black] (2.1, -1.45) circle (1.5pt);  % точка снизу справа
		\draw [fill=black] (-2.1, -1.45) circle (1.5pt); % точка снизу слева
		
		\draw (-0.5, 0.5) node {$\alpha$};
		\draw (0, 0) node {$\beta$};
		\draw (-1, -0.75) node {$d$};
		\draw (1, -0.75) node {$d$};
	\end{scriptsize}
\end{tikzpicture}
\end{center}
\end{figure}

\subsection{Обозначения}

d - диагональ трапеции \\
$\alpha, \beta$ - углы между диагоналями

\subsection{Формула площади равнобедренной трапеции через диагонали и угол между ними}

\[ S = \frac{d^2}{2}\sin\alpha = \frac{d^2}{2}\sin\beta \]

\newpage
\section{Формула площади равнобедренной трапеции через среднюю линию, боковую сторону и угол при основании}

\begin{figure}[h]
\begin{center}
	\begin{tikzpicture}

	\draw [shift={(-1, 1.45)}] (0, 0) -- (-109 : 0.25) arc (-109 : 0 : 0.25) -- cycle; % левый верхний угол
	\draw [shift={(1, 1.45)}] (0, 0) -- (180 : 0.25) arc (180 : 289 : 0.25) -- cycle;  % правый верхний угол

	\draw (-1, 1.45) -- (1, 1.45);       % верхняя линия
	\draw (1, 1.45) -- (2.1, -1.45);     % линия справа
	\draw (2.1, -1.45) -- (-2.1, -1.45); % нижняя линия
	\draw (-2.1, -1.45) -- (-1, 1.45);   % линия слева
	
	\draw (-1.55, 0) -- (1.55, 0); % средняя линия трапеции

	\draw [shift={(2.1, -1.45)}] (109 : 0.35) arc (109 : 180 : 0.35); % нижний правый угол
	\draw [shift={(2.1,-1.45)}] (109 : 0.3) arc (109 : 180 : 0.3);    % нижний правый угол

	\draw [shift={(-2.1, -1.45)}] (0 : 0.35) arc (0 : 70 : 0.35);     % нижний левый угол
	\draw [shift={(-2.1, -1.45)}] (0 : 0.3) arc (0 : 70 : 0.3);       % нижний левый угол

	\draw (-1.8, -0.2) -- (-1.45, -0.3);   % слева
	\draw (-1.85, -0.35) -- (-1.5, -0.45); % слева
	\draw (1.8, -0.2) -- (1.45, -0.3);     % справа
	\draw (1.85, -0.35) -- (1.5, -0.45);   % справа

	\begin{scriptsize}
		\draw [fill=black] (-1, 1.45) circle (1.5pt);    % точка вверху слева
		\draw [fill=black] (1, 1.45) circle (1.5pt);     % точка вверху справа
		\draw [fill=black] (2.1, -1.45) circle (1.5pt);  % точка снизу справа
		\draw [fill=black] (-2.1, -1.45) circle (1.5pt); % точка снизу слева
		
		\draw [fill=black] (-1.55, 0) circle (1.5pt); % точка в центре слева
		\draw [fill=black] (1.55, 0) circle (1.5pt);  % точка в центре справа

		\draw (-0.7, 1.1) node {$\alpha$};
		\draw (-1.6, -1.1) node {$\beta$};
		\draw (-1.8, 0) node {$с$};
		\draw (1.8, 0) node {$с$};
		\draw (0, -0.2) node {$m$};
	\end{scriptsize}
\end{tikzpicture}
\end{center}
\end{figure}

\subsection{Обозначения}

c - боковая сторона \\
m - средняя линия трапеции \\
$\alpha, \beta$ - углы между диагоналями

\subsection{Формула площади равнобедренной трапеции через среднюю линию, боковую сторону и угол при основании}

\[ S = mc\sin\alpha = mc\sin\beta \]

\newpage
\section{Формула площади равнобедренной трапеции через основания и высоту}

\begin{figure}[h]
\begin{center}
	\begin{tikzpicture}

	\draw [shift={(-1, 1.45)}] (0, 0) -- (-109 : 0.25) arc (-109 : 0 : 0.25) -- cycle; % левый верхний угол
	\draw [shift={(1, 1.45)}] (0, 0) -- (180 : 0.25) arc (180 : 289 : 0.25) -- cycle;  % правый верхний угол

	\draw (-1, 1.45) -- (1, 1.45);       % верхняя линия
	\draw (1, 1.45) -- (2.1, -1.45);     % линия справа
	\draw (2.1, -1.45) -- (-2.1, -1.45); % нижняя линия
	\draw (-2.1, -1.45) -- (-1, 1.45);   % линия слева
	
	\draw (-1, 1.45) -- (-1, -1.45); % высота трапеции

	\draw [shift={(2.1, -1.45)}] (109 : 0.35) arc (109 : 180 : 0.35); % нижний правый угол
	\draw [shift={(2.1,-1.45)}] (109 : 0.3) arc (109 : 180 : 0.3);    % нижний правый угол

	\draw [shift={(-2.1, -1.45)}] (0 : 0.35) arc (0 : 70 : 0.35);     % нижний левый угол
	\draw [shift={(-2.1, -1.45)}] (0 : 0.3) arc (0 : 70 : 0.3);       % нижний левый угол

	\draw (-1.7, 0) -- (-1.4, -0.08); % слева
	\draw (-1.68, 0.08) -- (-1.4, 0); % слева
	\draw (1.4, -0.08) -- (1.7, 0);   % справа
	\draw (1.4, 0) -- (1.7, 0.08);    % справа

	\begin{scriptsize}
		\draw [fill=black] (-1, 1.45) circle (1.5pt);    % точка вверху слева
		\draw [fill=black] (1, 1.45) circle (1.5pt);     % точка вверху справа
		\draw [fill=black] (2.1, -1.45) circle (1.5pt);  % точка снизу справа
		\draw [fill=black] (-2.1, -1.45) circle (1.5pt); % точка снизу слева
		
		\draw [fill=black] (-1, -1.45) circle (1.5pt);  % нижняя точка высоты трапеции

		\draw (0, 1.6) node {$a$};
		\draw (0, -1.65) node {$b$};
		\draw (-0.75, 0) node {$h$};
	\end{scriptsize}
\end{tikzpicture}
\end{center}
\end{figure}

\subsection{Обозначения}

a - нижнее основание \\
b - верхнее основание \\
h - высота трапеции

\subsection{Формула площади равнобедренной трапеции через основания и высоту}

\[ S = \frac{a + b}{a}h \]

\newpage
\section{Площадь треугольника по стороне и двум углам}

\begin{figure}[h]
\begin{center}
	\begin{tikzpicture}
	
		\draw  (0, 0) -- (1.5, 3); % (A, B)
		\draw  (1.5, 3) -- (4, 1); % (B, C)
		\draw  (4, 1) -- (0, 0);   % (A, C)
		
		\draw [shift={(0, 0)}] (0, 0) -- (14 : 0.6) arc (14 : 63.4 : 0.6) -- cycle; % угол A
		\draw [shift={(1.5, 3)}] (-116.5 : 0.6) arc (-116.5 : -38.6 : 0.6);         % угол В
		\draw [shift={(1.5, 3)}] (-116.5 : 0.5) arc (-116.5 : -38.6 : 0.5);         % угол В
		\draw [shift={(4, 1)}] (141.3 : 0.6) arc (141.3 : 194 : 0.6);               % угол С
		\draw [shift={(4, 1)}] (141.3 : 0.5) arc (141.3 : 194 : 0.5);               % угол С
		\draw [shift={(4, 1)}] (141.3 : 0.4) arc (141.3 : 194 : 0.4);               % угол С
		
	\begin{scriptsize}
		\draw [fill=black] (0, 0) circle (1.5pt);   % точка А
		\draw [fill=black] (1.5, 3) circle (1.5pt); % точка В
		\draw [fill=black] (4, 1) circle (1.5pt);   % точка С
		
		\draw (3, 2) node {$a$};
		\draw (2, 0.25) node {$b$};
		\draw (0.5, 1.5) node {$c$};
		
		\draw (0.75, 0.5) node {$\alpha$};
		\draw (1.65, 2.15) node {$\beta$};
		\draw (3.25, 1.15) node {$\gamma$};
	\end{scriptsize}
	\end{tikzpicture}
\end{center}
\end{figure}

\subsection{Обозначения}

a, b, c- стороны треугольника \\
$\alpha, \beta, \gamma$ - противолежащие углы

\subsection{Площадь треугольника через сторону и два угла}

\begin{align*}
S = \frac{a^2}{2} \times \frac{\sin(\beta)\sin(\gamma)}{\sin(\beta + \gamma)} = \frac{a^2}{2} \times \frac{\sin(\beta)\sin(\gamma)}{\sin(\alpha)} \\[6pt]
S = \frac{b^2}{2} \times \frac{\sin(\alpha)\sin(\gamma)}{\sin(\alpha + \gamma)} = \frac{b^2}{2} \times \frac{\sin(\alpha)\sin(\gamma)}{\sin(\beta)} \\[6pt]
S = \frac{c^2}{2} \times \frac{\sin(\alpha)\sin(\beta)}{\sin(\alpha + \beta)} = \frac{c^2}{2} \times \frac{\sin(\alpha)\sin(\beta)}{\sin(\gamma)} \\
\end{align*}

\newpage
\section{Формула площади правильного многоугольника}

\begin{figure}[h]
\begin{center}
	\begin{tikzpicture}
		\draw (0, 4) -- (-1.5, 3);
		\draw (-1.5, 3) -- (-1, 1);
		\draw (-1, 1) -- (1, 1);
		\draw (1, 1) -- (1.5, 3);
		\draw (1.5, 3) -- (0, 4);
		\draw (0, 4) -- (0, 2.5);
		\draw (-1.5, 3) -- (0, 2.5);
		\draw (1.5, 3) -- (0, 2.5);
		\draw (-1, 1) -- (0, 2.5);
		\draw (1, 1) -- (0, 2.5);
	\begin{scriptsize}
		\draw [fill=black] (0, 4) circle (1.5pt);
		\draw [fill=black] (-1.5, 3) circle (1.5pt);
		\draw [fill=black] (-1, 1) circle (1.5pt);
		\draw [fill=black](1, 1) circle (1.5pt);
		\draw [fill=black] (1.5, 3) circle (1.5pt);
		\draw [fill=black] (0, 2.5) circle (1.5pt);
		
		\draw (0.8, 3.75) node {$a$};
		\draw (1.5, 2) node {$a$};
		\draw (-0.8, 3.75) node {$a$};
		\draw (-1.5, 2) node {$a$};
		\draw (0, 0.8) node {$a$};		
	\end{scriptsize}
	\end{tikzpicture}
\end{center}
\end{figure}

\subsection{Обозначения}

a - сторона многоугольника \\
n - количество сторон

\subsection{Площадь правильного многоугольника}

\[ S = \frac{na^2}{4\tg \left(\dfrac{180\,^{\circ}}{n} \right)} \]

\newpage
\section{Площадь треугольника, формула Герона}

\begin{figure}[h]
\begin{center}
	\begin{tikzpicture}
		\draw  (0, 0) -- (1.5, 3); % (A, B)
		\draw  (1.5, 3) -- (4, 1); % (B, C)
		\draw  (4, 1) -- (0, 0);   % (A, C)
	\begin{scriptsize}
		\draw [fill=black] (0, 0) circle (1.5pt);   % точка А
		\draw [fill=black] (1.5, 3) circle (1.5pt); % точка В
		\draw [fill=black] (4, 1) circle (1.5pt);   % точка С
		
		\draw (3, 2) node {$a$};
		\draw (2, 0.25) node {$b$};
		\draw (0.5, 1.5) node {$c$};
	\end{scriptsize}
	\end{tikzpicture}
\end{center}
\end{figure}

\subsection{Обозначения}

a, b, c,- стороны треугольника \\[6pt]
p - полупериметр \\[6pt]
$p = \dfrac{( a + b + c)}{2}$

\subsection{Площадь правильного многоугольника}

\[ S = \sqrt{p(p - a)(p - b)(p - c)} \]

\newpage
\section{Формула расчета площади треугольника}

\begin{figure}[h]
\begin{center}
	\begin{tikzpicture}
		\draw (0, 0) -- (3, 3);
		\draw (3, 3) -- (5, 0);
		\draw (5, 0) -- (0, 0);
		\draw (3, 3) -- (3, 0);
	\begin{scriptsize}
		\draw [fill=black] (0, 0) circle (1.5pt);
		\draw [fill=black] (3, 3) circle (1.5pt);
		\draw [fill=black] (5, 0) circle (1.5pt);
		\draw [fill=black] (3, 0) circle (1.5pt);
		
		\draw (2,-0.3) node {$a$};
		\draw (2.75,1.5) node {$h$};
	\end{scriptsize}
	\end{tikzpicture}
\end{center}
\end{figure}

\subsection{Обозначения}

h - высота треугольника \\
a - основание

\subsection{Площадь правильного многоугольника}

\[ S = \frac{1}{2}ah \]

\end{document} % Конец текста.