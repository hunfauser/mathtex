\documentclass[a4paper,14pt]{article}

\usepackage{cmap}					% поиск в PDF
\usepackage{mathtext} 				% русские буквы в формулах
\usepackage[T2A]{fontenc}			% кодировка
\usepackage[utf8]{inputenc}			% кодировка исходного текста
\usepackage[english,russian]{babel}	% локализация и переносы

% Дополнительная работа с математикой
\usepackage{amsmath,amsfonts,amssymb,amsthm,mathtools} % AMS
\usepackage{icomma}                                    % "Умная" запятая

%% Шрифты
\usepackage{euscript} % Шрифт Евклид
\usepackage{mathrsfs} % Матшрифт

%% Перенос знаков в формулах (по Львовскому)
\newcommand*{\hm}[1]{#1\nobreak\discretionary{}
{\hbox{$\mathsurround=0pt #1$}}{}}

%%% Работа с картинками
\usepackage{graphicx}              % Для вставки рисунков
\usepackage{wrapfig}               % Обтекание рисунков текстом

%%% Работа с таблицами
\usepackage{array,tabularx,tabulary,booktabs} % Дополнительная работа с таблицами
\usepackage{longtable}                        % Длинные таблицы
\usepackage{multirow}                         % Слияние строк в таблице

\usepackage{pgf,tikz} % Работа с графикой
\usepackage{pgfplots}
\usepackage{pgfplotstable}

\usepackage[]{geometry}

\usetikzlibrary{arrows}
\newcommand{\degre}{\ensuremath{^\circ}}

%%% Заголовок
\author{Hun Fauser}
\title{Геометрия}
\date{\today}

\begin{document} % Конец преамбулы, начало текста.

\maketitle

\newpage
\section{Площадь круга}

\begin{figure}[h]
\begin{center}
	\begin{tikzpicture}
		\draw (0, 0) circle (2.5);   % Нарисовать круг
		\draw (0, -2.5) -- (0, 2.5); % Нарисовать вертикальную линию
		\draw (0, 0) -- (2.5, 0);    % Нарисовать горизонтальную линию
	\begin{scriptsize}
		\draw (-0.25, -0.25) node {$D$};
		\draw (1, -0.25) node {$r$};
	\end{scriptsize}
	\end{tikzpicture}
	\caption{Зная диаметр или радиус круга, можно найти его площадь.}
\end{center}
\end{figure}

\subsection{Обозначения}

r = радиус круга \\
R = диаметр \\
$\pi \approx 3.14$

\subsection{Площадь круга}

$S = \pi r^2 = \frac{\pi}{4}D^2$

\subsection{Периметр круга}

$p = 2\pi r = \pi D$

\newpage
\section{Площадь прямоугольника}

\begin{figure}[h]
	\begin{center}
\begin{tikzpicture}[auto]
	\draw (0, 0) -- (0, 3); %(A, B)
	\draw (0, 0) -- (5, 0); %(B, C)
	\draw (5, 0) -- (5, 3); %(C, D)
	\draw (5, 3) -- (0, 3); %(A, D)
\begin{scriptsize}
\draw (-0.25, 3.25) node {$A$};
\draw (-0.25, -0.25) node {$B$};
\draw (5.25, -0.25) node {$C$};
\draw (5.25, 3.25) node {$D$};

\draw (-0.25, 1.5) node {$a$};
\draw (2.5, -0.25) node {$b$};
\draw (5.25, 1.5) node {$c$};
\draw (2.5, 3.25) node {$d$};
\end{scriptsize}
\end{tikzpicture}
\end{center}
\end{figure}

\subsection{Обозначения}

b, d = длина прямоугольника \\
a, c = ширина прямоугольника

\subsection{Площадь прямоугольника}

$S = ab$

\newpage
\section{Площадь эллипса}

\begin{figure}[h]
\begin{center}
	\begin{tikzpicture}
		\draw (0,0) ellipse (3cm and 2cm); % Нарисовать эллипс
		\draw (0, 0) -- (0, 2);     	   % Нарисовать вертикальную линию
		\draw (0, 0) -- (3, 0);    		   % Нарисовать горизонтальную линию
		\draw[dashed] (-3, 0) -- (0, 0);   % Нарисовать горизонтальную пунктирную линию
		\draw[dashed] (0, 0) -- (0, -2);   % Нарисовать вертикальную пунктирную линию
	\begin{scriptsize}
		\draw (1.5, -0.25) node {$R$};
		\draw (-0.25, 1) node {$r$};
	\end{scriptsize}
	\end{tikzpicture}
\end{center}
\end{figure}

\subsection{Обозначения}

R = большая полуось \\
r = малая полуось

\subsection{Площадь прямоугольника}

$S = \pi Rr$

\newpage
\section{Формула площади равнобедренной трапеции через стороны и угол}

\begin{figure}[h]
\begin{center}
	\begin{tikzpicture}
		\draw (0, 0) -- (5, 0);     %(B, C)
		\draw (0, 0) -- (1, 2.5);   %(B, A)
		\draw (1, 2.5) -- (4, 2.5); %(A, D)
		\draw (4, 2.5) -- (5, 0);   %(D, C)
	\begin{scriptsize}
		\draw (0.75, 2.5) node {$A$};
		\draw (-0.25, 0) node {$B$};
		\draw (5.25, 0) node {$C$};
		\draw (4.25, 2.5) node {$D$};
		
		\draw[fill=black] (0, 0) circle (1.5pt);
		\draw[fill=black] (5, 0) circle (1.5pt);
		\draw[fill=black] (1, 2.5) circle (1.5pt);
		\draw[fill=black] (4, 2.5) circle (1.5pt);
		
		\draw (2.5, -0.25) node {$a$};
		\draw (2.5, 2.75) node {$b$};
		\draw (0.25, 1.25) node {$c$};
		\draw (4.75, 1.25) node {$d$};
		
		\draw [shift={(1, 2.5)}] (0,0) -- (-111.5 : 0.5) arc (-111.5 : 0 : 0.5) -- cycle; % Угол А
		\draw [shift={(4, 2.5)}] (0,0) -- (180 : 0.5) arc (185 : 290 : 0.5) -- cycle; % Угол D
		
		\draw [shift={(5.0,0.0)}] (111 : 0.65) arc (112 : 180 : 0.65); % Угол C
		\draw [shift={(5.0,0.0)}] (112 : 0.55) arc (112 : 180 : 0.55); % Угол C
		\draw [shift={(0.0,0.0)}] (0 : 0.65) arc (0 : 68 : 0.65);       % Угол D
		\draw [shift={(0.0,0.0)}] (0 : 0.55) arc (0 : 68 : 0.55);       % Угол D
	\end{scriptsize}
	\end{tikzpicture}
\end{center}
\end{figure}

\subsection{Обозначения}

R = большая полуось \\
r = малая полуось

\subsection{Площадь прямоугольника}

\end{document} % Конец текста.