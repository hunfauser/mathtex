\documentclass[a4paper,12pt]{article}

\usepackage{cmap}					% поиск в PDF
\usepackage{mathtext} 				% русские буквы в формулах
\usepackage[T2A]{fontenc}			% кодировка
\usepackage[utf8]{inputenc}			% кодировка исходного текста
\usepackage[english,russian]{babel}	% локализация и переносы

% Дополнительная работа с математикой
\usepackage{amsmath,amsfonts,amssymb,amsthm,mathtools} % AMS
\usepackage{icomma}                                    % "Умная" запятая

% Шрифты
\usepackage{euscript} % Шрифт Евклид
\usepackage{mathrsfs} % Красивый матшрифт

% Убрать вертикальный отступ у списков
\usepackage{enumitem}
\setlist{nolistsep}

\usepackage{graphicx}
\usepackage{hyperref}
\usepackage[usenames,dvipsnames,svgnames,table,rgb]{xcolor}
\hypersetup{				                                  % Гиперссылки
	unicode=true,                                            % русские буквы в раздела PDF
	pdftitle={Заголовок},                                % Заголовок
	pdfauthor={Автор},                                   % Автор
	pdfsubject={Тема},                                    % Тема
	pdfcreator={Создатель},                          % Создатель
	pdfproducer={Производитель},              % Производитель
	pdfkeywords={keyword1} {key2} {key3}, % Ключевые слова
	colorlinks=true,       	                                 % false: ссылки в рамках; true: цветные ссылки
	linkcolor=red,                                            % внутренние ссылки
	citecolor=green,                                        % на библиографию
	filecolor=magenta,                                    % на файлы
	urlcolor=blue                                             % на URL
}

\title{Statistika}
\author{Hun Fauser}
\date{\today}

\begin{document} % Конец преамбулы, начало текста.

\maketitle

\section{LEKCE 01}

Pravděpodobnost jevu A (P (A)) je rovna poměru počtu případů jevu A (m) 
k počtu případu možných (n). $P\in<0;1>$

\[  P_{A} = \frac{m}{n} \]

Ke každému náhodnému jevu A existuje opačný jev \={A}, který je nenastoupením jevu A, platí:

\[  P (\overline{A}) = 1 - P(A) \]

PŘÍKLAD

S jakou pravděpodobností padne po jednom hodu kostkou číslo 6? \\

ŘEŠENÍ \\

$P = \dfrac{1}{6}$, respektive P = 1.167 \\

\Large \textbf{Statistická definice pravděpodobnosti}

\normalsize
Statistická definice pravděpodobnosti je založená na pojmu relativní četnosti, 
kdy pravděpodobnost chápeme jako kvantitativní vlastnost náhodného jevu, 
kterou měříme pomocí relativní četnosti výskytu náhodného jevu, přičemž 
přesnost měření roste s počtem opakování nezávislých pokusů.

PŘÍKLAD

Určitý pokus (hod kostkou) může být mnohonásobně (N-krát) opakovan 
za neměnných podmínek - N pokusů. Jestliže se u N pokusů vyskytne jev A (padne číslo 6)
M-krát, poměr $\dfrac{M}{N}$ je relativní četností jevu A.

\newpage
\Large \textbf{Seminar}

\normalsize
Matematika - z řeckého jazyka

Matematika je věda zabývající se z formálního hlediska kvantitou, 
strukturou, prostorem a změnou. \\

Charakteristickou vlastnosti matematiky je její důraz 
na absolutní přesnost metod a nezpochybnitelnost výsledků \\

Historie matematiky sahá až do pravěku, kdy vznikly první abstraktní 
matematické pojmy - čisla. \\

Nula - nepřirozené číslo

Eukleides z Megary - řecký matematik, 4-3st. před n.l. Zakladatel matematiky.

Leonardo Fibonacci - arabské čislice. 13st.

Matematika je univerzální jazyk.

Existuje jenom jedna forma matematiky. \\

Elementární matematika
\begin{itemize}
	\item Operace s čísly
	\item Rovnice
	\item Geometrické objekty \\
\end{itemize}

Aplikovaná matematika
\begin{itemize}
	\item Ekonomie
	\item Statistika
	\item Chemie \\
\end{itemize}

Čistá matematika - vysoce abstraktní věda. Nepraktická.

V matematice se využivá řecká abeceda a symboly.

Konstanta - dané čislo. Nemusí být známá.

Opakem konstanty je proměnná

\begin{align*}
\pi = 3.141592654 \\
e = 2.718281828 \\
\phi = 1.618033989 \\
\end{align*}

\newpage
\textbf{Eulerovo číslo}

Eulerovo číslo jako limita následující posloupnosti:
\[
	e = \lim_{n\to\infty} \left(1 + \frac{1}{n} \right)^n
\]

Eulerovo číslo jako součet následující nekonečné řady.
\[
e = \sum_{n=0}^{\infty} \frac{1}{n!} = \frac{1}{0!} + \frac{1}{1!} + \frac{1}{2!} + \frac{1}{3!} + \frac{1}{4!} + ...
\]

Eulerovo číslo jako jediné číslo x > 0, pro které platí, že:
\[
\ln x = \displaystyle\int_{1}^{x} \frac{dt}{t} = 1
\]

Výrok
\begin{itemize}
	\item Je tokové tvrzení, o kterém lze jednoznačně rozhodnout, zda je to pravdivě nebo ne
	\item Označuje se malými pismeny latinské abecedy
	\item Jeho platnost se označuje číslem 1
	\item Neplatnost 0 \\
\end{itemize}

Výroky
\begin{itemize}
	\item Jednoduché (nedelitelné)
	\item Složené \\
\end{itemize}

Jednoduché výroky
\begin{itemize}
	\item 3<5 (1) - platí
	\item 2>3 (0) - neplatí \\
\end{itemize}

Symbol negace - —

Slovně "není pravda" \\

Logické operace
\begin{align*}
Konjunkce~a \wedge b \\
Disjunkce~a \vee b \\
Implikace~a \Rightarrow b \\
Ekvivalence~ a \Leftrightarrow	b
\end{align*}

Výroková forma - vyraz, který obsahuje jednu nebo vice proměnných z 
daného, předem určeného souboru, který se po dosažení 
konkretních hodnot proměnných stane výrokem.

\newpage
Kvantifikatory
\begin{itemize}
	\item Velký
	\item Malý \\
\end{itemize}

Obecný
\begin{itemize}
	\item Daná vlastnost platí pro každý prvek
	\item Výroky, které obsahují obecný kvantifikator se nazývají obecné výroky \\
\end{itemize}

\begin{table}[h]
	\centering
	\resizebox{0.7\textwidth}{!}{%
		\begin{tabular}{|l|l|}
			\hline
			\textbf{Výrok}   & \textbf{Negace výroku} \\ \hline
			Každý...je       & Aspoň jeden...není     \\ \hline
			Aspoň jeden...je & Žádný...není           \\ \hline
			Aspoň n...je     & Nejvýše...je           \\ \hline
			Nejvýše n...je   & Aspoň...je             \\ \hline
			Právě n...je     &                        \\ \hline
		\end{tabular}
	}
\end{table}

\textbf{Statistika}

Status - stav
\begin{itemize}
	\item Čiselné udaje
	\item Praktická činnost
	\item Vědní disciplina \\
\end{itemize}

\href{http://czso.cz}{Český Statistický Úřad} \\

Statistické charakterisky
\begin{itemize}
	\item Miry polohy
	\item Miry variability (s nedefinovanou proměnlivostí uvnitř souboru dat, s definovanou proměnlivostí uvnitř souboru dat) \\
\end{itemize}

\begin{align*}
a(x) : (x + 1)^2 = x^2 + 2x+1; x \in R \\
\forall	((x^2 - 1) - (x + 1)(x - 1)); x \in R \\
\exists	(x^2 + 1 = 0); x \in R \\
f(x) = \frac{1}{\sigma \sqrt{2\pi}} e^{-\frac{(x - \mu)^2}{2 \sigma^2}}
\end{align*}

\newpage
Aritmetický průměr $\overline{x}$ střední hodnota kvantitativního
statistického znaku 

(součet hodnot, dělený jejích počtem)

\[
	\overline{x} = \frac{\sum x_{i}}{n}
\]

, kde \textbf{n} je rozsah souboru \\

Modus \^{x} - hodnota nejčastěji se v souboru vyskytující

Median \~{x} = prostřední hodnota  $\Rightarrow$ je-li n(rozsah souboru) 
liché číslo, median je prostřední hodnota, je-li n sudé číslo
je median aritmetickým průměrem dvou prostředních hodnot. \\

100P\% kvantil $x_{p}$ je číslo, které odděluje 100P\% nejmenších hodnot
náhodné veličiny X.

Tedy 50\% kvantil $x_{0.50}$ je totež co medián \\

Dobrý popis rozdělení pravděpodobnosti dostaneme stanovením dostatečného počtu kvantilů.

Kvantily zaznamenané po dvaceti pěti procentech nazýváme kvartily,
po deseti procentech decily a po jednom procentu percentily.

Tedy 25\% kvantil je 1. kvartil (dolní kvartil), 10\% kvantil je 1. decil a podobně 1\% 
kvantil je 1. percentil

Medián je totéž co 50\% kvantil, 2. kvartil, 5. decil nebo 50. percentil.

S použitím kvartilů. decilů a percentilů se často setkáváme při prezentaci výsledků
antropometrických studií.
 














\newpage
\section{MATH TWEAKS}


\subsection{Number 01}
\begin{align*}
	A &= 100\% ~~~~~
	B = x \\
	x &= \frac{B \times 100}{A}
\end{align*}

\subsection{Number 02}
\begin{align*}
A &= 100\% ~~~~~
x = y\% \\
x &= \frac{A}{B} \times y
\end{align*}

\subsection{Number 03}
\begin{align*}
A + B &= 110 \\
A &= B + 100 \\
(вместо~А~подставляем~B &+ 100~в~первое~уравнение) \\
B + 100 + B &= 110 \\
2B &= 10 \\
B = \frac{10}{2} &= 5 \\
(подставляем~ответ~&B~во~второе~уравнение) \\
A = 5 + 100 &= 105 \\
Ответ:
B = 5; A &= 105;
\end{align*}

\end{document} % Конец текста.