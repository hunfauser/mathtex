\documentclass[a4paper,12pt]{article}

\usepackage{cmap}					% поиск в PDF
\usepackage{mathtext} 				% русские буквы в формулах
\usepackage[T2A]{fontenc}			% кодировка
\usepackage[utf8]{inputenc}			% кодировка исходного текста
\usepackage[english,russian]{babel}	% локализация и переносы

% Дополнительная работа с математикой
\usepackage{amsmath,amsfonts,amssymb,amsthm,mathtools} % AMS
\usepackage{icomma}                                    % "Умная" запятая

% Шрифты
\usepackage{euscript} % Шрифт Евклид
\usepackage{mathrsfs} % Красивый матшрифт

\title{Statistika}
\author{Hun Fauser}
\date{\today}

\begin{document} % Конец преамбулы, начало текста.

\maketitle

\section{LEKCE 01}

Pravděpodobnost jevu A (P (A)) je rovna poměru počtu případů jevu A (m) 
k počtu případu možných (n). $P\in<0;1>$

\[  P_{A} = \frac{m}{n} \]

Ke každému náhodnému jevu A existuje opačný jev \={A}, který je nenastoupením jevu A, platí:

\[  P (\overline{A}) = 1 - P(A) \]

PŘÍKLAD

S jakou pravděpodobností padne po jednom hodu kostkou číslo 6? \\

ŘEŠENÍ \\

$P = \dfrac{1}{6}$, respektive P = 1.167 \\

\Large \textbf{Statistická definice pravděpodobnosti}

\normalsize
Statistická definice pravděpodobnosti je založená na pojmu relativní četnosti, 
kdy pravděpodobnost chápeme jako kvantitativní vlastnost náhodného jevu, 
kterou měříme pomocí relativní četnosti výskytu náhodného jevu, přičemž 
přesnost měření roste s počtem opakování nezávislých pokusů.

PŘÍKLAD

Určitý pokus (hod kostkou) může být mnohonásobně (N-krát) opakovan 
za neměnných podmínek - N pokusů. Jestliže se u N pokusů vyskytne jev A (padne číslo 6)
M-krát, poměr $\dfrac{M}{N}$ je relativní četností jevu A.













\end{document} % Конец текста.