\documentclass[a4paper,12pt]{article}

\usepackage{cmap}					% поиск в PDF
\usepackage{mathtext} 				% русские буквы в формулах
\usepackage[T2A]{fontenc}			% кодировка
\usepackage[utf8]{inputenc}			% кодировка исходного текста
\usepackage[english,russian]{babel}	% локализация и переносы

% Дополнительная работа с математикой
\usepackage{amsmath,amsfonts,amssymb,amsthm,mathtools} % AMS
\usepackage{icomma} % "Умная" запятая

%% Шрифты
\usepackage{euscript}	 % Шрифт Евклид
\usepackage{mathrsfs}    % Матшрифт

%%% Заголовок
\author{Hun Fauser}
\title{Sbírka příkladů z matematiky k přijímacím zkouškám na VŠ - решебник}
\date{\today}

\begin{document} % Конец преамбулы, начало текста.

\maketitle

\section{Algebraické výrazy a jejich úpravy}

\subsection{} % задания 1(a-c)

\begin{align*}
a^2 \times a^4 \times a^3 
= a^{2 + 4 + 3} 
&= a^9 \\[6pt]
%===%
\frac{b^5 \times b^6}{b^2} 
= b^{5 + 6 - 2} 
&= b^9 \\[6pt]
%===%
(c^{-3})^9 
= c^{-3 \times 9} 
= c^{-27} 
&= \frac{1}{c^{27}}
\end{align*}

\subsection{} % задания 2(a-c)

\begin{align*}
%===%
(2x^3)^4 + (x^7)^2 \times x^{-2} 
= 16x^{12} + x^{14} \times x^{-2} 
= 16x^{12} + x^{12} 
&= 17x^{12} \\[6pt]
%===%
(3x^2)^3 x^4 - (2x^2)^5 
= 27x^6 \times x^4 - 32x^{10} 
= 27x^{10} - 32 x^{10} 
&= -5x^{10} \\[6pt]
%===%
(x^6)(-2x^{-3})^4 + (4x^{-3})^2 
= (x^6) \left( \frac{16}{x^{12}} \right) + \frac{16}{x^6} 
= \frac{16}{x^{12 - 6}} + \frac{16}{x^6} 
&= \frac{32}{x^{6}}
\end{align*}

\subsection{} % задания 3(a-c)

\begin{align*}
\frac{a^0 \times a^{2n - 5}}{a^{1 - n}} 
= a^{2n - 5 - 1 + n} 
&= a^{3n - 6} \\[6pt]
%===%
\frac{a^{2n + 3} \times a^{3n - 2}}{(a^n)^2} 
= a^{2n + 3 + 3n - 2 - 2n} 
&= a^{3n + 1} \\[6pt]
%===%
\frac{a^{1 - 4n} \times (a^{n + 2})^2}{a^{3n - 1}} 
= \frac{a^{1 - 4n} \times a^{2n + 4}}{a^{3n - 1}} 
= a^{1 - 4n + 2n + 4 - 3n + 1} 
&= a^{6 - 5n} 
\end{align*}

\subsection{} % задания 4(a-e)

\begin{align*}
\frac{18^4 \times 27^{-4} \times 49^2}{14^3} 
= 3^8 \times 2^4 \times 3^{-12} \times 7^4 \times 7^{-3} \times 2^{-3} 
= 3^{8 - 12} \times 2^{4 - 3} \times 7^{4 - 3} 
= \frac{2 \times 7}{3^4} 
&= \frac{14}{81} \\[6pt]
%===%
\frac{20^{-3} \times 38^0 \times 100^2}{16^{-2} \times 2^5} 
= 10^{-3} \times 2^{-3} \times 10^4 \times 2^8 \times 2^{-5} 
= 10^{-3 + 4} \times 2^{-3 + 8 - 5} 
= 10 \times 1 
&= 10 \\[6pt]
%=====%
\frac{88^3 \times 55^{-2} \times 10000^{-1}}{25^{-3} \times 22} 
= 2^9 \times 11^3 \times 5^{-2} \times 11^{-2} \times 5^{-4} \times 2^{-4} \times 5^6 \times 2^{-1} \times 11^{-1} = \\[6pt]
= 2^{9 - 4 - 1} \times 11^{3 - 2 - 1} \times 5^{-2 - 4 + 6}
= 2^4 \times 11^0 \times 5^0
&= 16 \\[6pt]
%=====%
\frac{(-4)^3}{3^2} \div \frac{(-2)^4}{3^5}
= (-2)^6 \times 3^{-2} \times 3^5 \times (-2)^{-4}
= (-2)^{6 - 4} \times 3^{-2 + 5} 
= 4 \times 27
&= 108  \\[6pt]
%=====%
\frac{2^{-6} \times 3^{-2} \times 5^3}{(2^{-4} \times 5^3)^2 \div 3^4}
= \frac{2^{-6} \times 3^{-2} \times 5^3 \times 3^4}{2^{-8} \times 5^6}
= \frac{2^8 \times 5^3 \times 3^4}{2^6 \times 5^6 \times 3^2} = \\[6pt]
= 2^{8 - 6} \times 5^{3 - 6} \times 3^{4 - 2}
= 4 \times \frac{1}{125} \times 9
&= \frac{36}{125}
%=====%
\end{align*}

\subsection{} % задания 5(a-с)

\begin{align*}
\left( \frac{a^{-2}b^4c^{-1}}{d^3} \right)^{-3} \div \left( \frac{d^4}{c^5} \right)^{-2} 
= \frac{d^9}{a^{-6}b^{12}c^{-3}} \times \frac{d^8}{c^{10}}
&= \frac{a^6 d^{17}}{b^{12} c^7} \\[6pt]
%===%
\left( \frac{a^2 c^3}{b^4 d^0} \right)^{-2} \div \left( \frac{b^4}{d^{-3}} \right)^3
= \frac{b^8}{a^4 c^6} \times \frac{d^{-9}}{b^{12}}
= \frac{b^{-4} d^{-9}}{a^4 c^6}
&= \frac{1}{a^4 b^4 c^6 d^9} \\[6pt]
%===%
\left( \frac{b^3}{a^4b^{-1}c^2} \right)^{-4} \div \left( \frac{a^2b^{-3}}{a^{-4}c^2} \right)^{-2}
= \left( \frac{a^{16}b^{-4}c^8}{b^{12}} \right) \times \left( \frac{a^4b^{-6}}{a^{-8}c^4} \right)
= \frac{a^{16 + 4 + 8} \times c^{8 - 4}}{b^{12 + 4 + 6}}
&= \frac{a^{28}c^4}{b^{22}}
\end{align*}

\subsection{} % задания 6(a-f)

\begin{align*}
\sqrt{2} - \sqrt{18} + \sqrt{32} + \sqrt{98}
= \sqrt{2} - 3\sqrt{2} + 4\sqrt{2} + 7\sqrt{2} = \\[6pt]
= (1 + 3 + 4 + 7)\sqrt{2}
&= 15\sqrt{2} \\[6pt] % ПРОВЕРИТЬ
%===%
\sqrt{175} - (\sqrt{8} - \sqrt{7} + \sqrt{28})
= 5\sqrt{7} - 2\sqrt{2} + \sqrt{7} - 2\sqrt{7} = \\[6pt]
= (5 + 1 - 2)\sqrt{7} - 2\sqrt{2}
&= 4\sqrt{7} - 2\sqrt{2} \\[6pt]
%===%
\sqrt{90} + \sqrt{160} - (\sqrt{63} + \sqrt{10})
= 3\sqrt{10} + 4\sqrt{10} - 3\sqrt{7} - \sqrt{10} = \\[6pt]
= (3 + 4 - 1)\sqrt{10} - 3\sqrt{7}
&= 6\sqrt{10} - 3\sqrt{7} \\[6pt]
%===%
2 \times 8^{\frac{1}{2}} - 7 \times 8^{\frac{1}{2}} + 5 \times 72^{\frac{1}{2}} - 50^{\frac{1}{2}}
= 4\sqrt{2} - 14\sqrt{2} + 30\sqrt{2} - 5\sqrt{2} = \\[6pt]
= (4 - 14 + 30 - 5)\sqrt{2}
&= 15\sqrt{2} \\[6pt]
%===%
2^{\frac{1}{2}} \times 4^{\frac{1}{3}} \times 8^{\frac{1}{4}} \times 16^{\frac{1}{6}} \times 32^{\frac{1}{12}}
= 2^{\frac{1}{2}} \times 2^{\frac{2}{3}} \times 2^{\frac{3}{4}} \times 2^{\frac{4}{6}} \times 2^{\frac{5}{12}}
= 2^{\frac{6 + 8 + 9 + 8 + 5}{12}}
= 2^{\frac{36}{12}}
= 2^{3}
&= 8 \\[6pt]
%===%
[2 \times (2 \times 2^{\frac{1}{2}})^{\frac{1}{2}}]^{\frac{1}{2}}
= 2^{\frac{1}{2}} \times 2^{\frac{1}{4}} \times 2^{\frac{1}{8}}
= 2^{\frac{4 + 2 + 1}{8}}
= \sqrt[8]{2^7}
\end{align*}

\subsection{} % задания 7(a-c)

\begin{align*}
\sqrt[3]{54} + \sqrt[3]{16} - \sqrt[3]{2} + \sqrt[3]{64}
= 3\sqrt[3]{2} + 2\sqrt[3]{2} - \sqrt[3]{2} + 4 = \\[6pt]
= (3 + 2 - 1)\sqrt[3]{2} + 4 
&= 4\sqrt[3]{2} + 4 \\[6pt]
%===%
\sqrt[3]{24} - (\sqrt[3]{3} - \sqrt[3]{81}) - \sqrt[3]{9}
= 2\sqrt[3]{3} - \sqrt[3]{3} + 3\sqrt[3]{3} - \sqrt[3]{9} = \\[6pt]
= (2 - 1 + 3)\sqrt[3]{3} - \sqrt[3]{9}
&= 4\sqrt[3]{3} - \sqrt[3]{9} \\[6pt]
%===%
\sqrt[3]{250} - (\sqrt[3]{40} + \sqrt[3]{16}) + \sqrt[3]{135}
= 5\sqrt[3]{2} - 2\sqrt[3]{5} - 2\sqrt[3]{2} + 3\sqrt[3]{5} = \\[6pt]
= (5 - 2)\sqrt[3]{2} + (-2 + 3)\sqrt[3]{5}
&= 3\sqrt[3]{2} + \sqrt[3]{5}
\end{align*}

\subsection{} % задания 8(a-d)

\begin{align*}
\sqrt{x} \times \sqrt[3]{x} \times \sqrt[7]{x}
= x^{\frac{1}{2} + \frac{1}{3} + \frac{1}{7}}
= x^{\frac{21 + 14 + 6}{42}}
= x^{\frac{41}{42}}
&= \sqrt[42]{x^{41}} \\[6pt]
%===%
\sqrt{\frac{\sqrt[3]{x}}{\sqrt[6]{x}}}
= x^{\frac{1}{6} - \frac{1}{12}}
= x^{\frac{2 - 1}{12}}
&= \sqrt[12]{x} \\[6pt]
%===%
\frac{\sqrt[4]{x \times \sqrt[3]{x}}}{\sqrt[3]{x^4 \times \sqrt{x}}}
= \frac{x^{\frac{1}{4}} \times x^{\frac{1}{12}}}{x^{\frac{4}{3}} \times x^{\frac{1}{3}}} 
= x^{\frac{3 + 1 - 16 - 4}{12}}
= x^{\frac{-16}{12}}
&= \frac{1}{\sqrt[3]{x^4}} \\[6pt]
%===%
\sqrt{a \times \sqrt{a \times \sqrt{a \times \sqrt{a}}}}
= a^{\frac{1}{2}} \times a^{\frac{1}{4}} \times a^{\frac{1}{8}} \times a^{\frac{1}{16}}
= a^{\frac{8 + 4 + 2 + 1}{16}}
= a^{\frac{15}{16}}
&= a\sqrt[16]{a^{15}}
\end{align*}






\end{document} % Конец текста.


% \left( \frac{}{} \right)
% ^{\frac{}{}}