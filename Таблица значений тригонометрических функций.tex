\documentclass[landscape,12pt]{article}

\usepackage{cmap}					% поиск в PDF
\usepackage{mathtext} 				% русские буквы в формулах
\usepackage[T2A]{fontenc}			% кодировка
\usepackage[utf8]{inputenc}			% кодировка исходного текста
\usepackage[english,russian]{babel}	% локализация и переносы

% Дополнительная работа с математикой
\usepackage{amsmath,amsfonts,amssymb,amsthm,mathtools} % AMS
\usepackage{icomma}                                    % "Умная" запятая

%% Шрифты
\usepackage{euscript} % Шрифт Евклид
\usepackage{mathrsfs} % Красивый матшрифт

% Работа с таблицами
\usepackage{graphicx}
\usepackage[table,xcdraw]{xcolor}
\usepackage{multirow}

%%% Заголовок
\author{Hun Fauser}
\title{Таблица значений тригонометрических функций}
\date{\today}

\begin{document} % Конец преамбулы, начало текста.

\maketitle

\renewcommand{\arraystretch}{1.5}
\begin{table}[h]
\centering
\resizebox{\textwidth}{!}{%
\begin{tabular}{|c|c|c|c|c|c|c|c|c|c|c|c|c|c|c|c|c|c|}
\hline
\multirow{2}{*}{} & 0 & $\frac{\pi}{6}$ & $\frac{\pi}{4}$ & $\frac{\pi}{3}$ & $\frac{\pi}{2}$ & $\frac{2\pi}{3}$ & $\frac{3\pi}{4}$ & $\frac{5\pi}{6}$ & $\pi$ & $\frac{7\pi}{6}$ & $\frac{5\pi}{4}$ & $\frac{4\pi}{3}$ & $\frac{3\pi}{2}$ & $\frac{5\pi}{3}$ & $\frac{7\pi}{4}$ & $\frac{11\pi}{6}$ & $2\pi$ \\ \cline{2-18}
 & 0 & $30\,^{\circ}$ & $45\,^{\circ}$ & $60\,^{\circ}$ & $90\,^{\circ}$ & $120\,^{\circ}$ & $135\,^{\circ}$ & $150\,^{\circ}$ & $180\,^{\circ}$ & $210\,^{\circ}$ & $225\,^{\circ}$ & $240\,^{\circ}$ & $270\,^{\circ}$ & $300\,^{\circ}$ & $315\,^{\circ}$ & $330\,^{\circ}$ & $360\,^{\circ}$ \\ \hline
$\sin\alpha$ & 0 & $\frac{1}{2}$ & $\frac{\sqrt{2}}{2}$ & $\frac{\sqrt{3}}{2}$ & 1 & $\frac{\sqrt{3}}{2}$ & $\frac{\sqrt{2}}{2}$ & $\frac{1}{2}$ & 0 & $-\frac{1}{2}$ & $-\frac{\sqrt{2}}{2}$ & $-\frac{\sqrt{3}}{2}$ & -1 & $-\frac{\sqrt{3}}{2}$ & $-\frac{\sqrt{2}}{2}$ & $-\frac{1}{2}$ & 0 \\ \hline
$\cos\alpha$ & 1 & $\frac{\sqrt{3}}{2}$ & $\frac{\sqrt{2}}{2}$ & $\frac{{1}}{2}$ & 0 & $-\frac{1}{2}$ & $-\frac{\sqrt{2}}{2}$ & $-\frac{\sqrt{3}}{2}$ & -1 & $-\frac{\sqrt{3}}{2}$ & $-\frac{\sqrt{2}}{2}$ & $-\frac{1}{2}$ & 0 & $\frac{1}{2}$ & $\frac{\sqrt{2}}{2}$ & $\frac{\sqrt{3}}{2}$ & 1 \\ \hline
$\tg\alpha$ & 0 & $\frac{1}{\sqrt{3}}$ & 1 & $\sqrt{3}$ & \textemdash & $-\sqrt{3}$ & -1 & $-\frac{1}{\sqrt{3}}$ & 0 & $\frac{1}{\sqrt{3}}$ & 1 & $\sqrt{3}$ & \textemdash & $-\sqrt{3}$ & -1 & $-\frac{1}{\sqrt{3}}$ & 0 \\ \hline
$\ctg\alpha$ & \textemdash & $\sqrt{3}$ & 1 & $\frac{1}{\sqrt{3}}$ & 0 & $-\frac{1}{\sqrt{3}}$ & -1 & $-\sqrt{3}$ & \textemdash & $-\sqrt{3}$ & 1 & $\frac{1}{\sqrt{3}}$ & 0 & $-\frac{1}{\sqrt{3}}$ & -1 & $-\sqrt{3}$ & \textemdash \\ \hline
\end{tabular}
}
\end{table}

\end{document} % Конец текста.