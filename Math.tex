\documentclass[a4paper,12pt]{article}

\usepackage{cmap}					% поиск в PDF
\usepackage{mathtext} 				% русские буквы в формулах
\usepackage[T2A]{fontenc}			% кодировка
\usepackage[utf8]{inputenc}			% кодировка исходного текста
\usepackage[english,russian]{babel}	% локализация и переносы

% Дополнительная работа с математикой
\usepackage{amsmath,amsfonts,amssymb,amsthm,mathtools} % AMS
\usepackage{icomma} % "Умная" запятая

%% Шрифты
\usepackage{euscript}	 % Шрифт Евклид
\usepackage{mathrsfs}    % Красивый матшрифт

%% Перенос знаков в формулах (по Львовскому)
\newcommand*{\hm}[1]{#1\nobreak\discretionary{}
{\hbox{$\mathsurround=0pt #1$}}{}}

%%% Заголовок
\author{Hun Fauser}
\title{Математическая шпаргалка}
\date{\today}

\begin{document} % Конец преамбулы, начало текста.

\maketitle

\section{Формулы сокращенного умножения}

\begin{align*}
(a + b)^2 &= a^2 + 2ab + b^2 \\
(a - b)^2 &= a^2 - 2ab + b^2 \\
a^2 - b^2 &= (a - b)(a + b) \\
a^3 - b^3 &= (a - b)(a^2 + ab + b^2) \\
a^3 + b^3 &= (a + b)(a^2 - ab + b^2) \\
(a + b)^3 &= a^3 + 3a^2b + 3ab^2 + b^3 \\
(a - b)^3 &= a^3 - 3a^2b + 3ab^2 - b^3
\end{align*}

\section{Свойства степеней и корней}

\begin{align*}
a^0 &= 1 ~(a \ne 0) \\
a^{-n} &= 1/a^n ~(a > 0, n ~\varepsilon ~Q) \\
a^n \times a^m &= a^{n + m} \\
\left(\frac{a^n}{a^m}\right) &= a^{n - m} (a \ne 0) \\
(a^n)^m &= a^{n \times m} \\
(ab)^n &= a^nb^n \\
\left(\frac{a}{b}\right)^n &= \frac{a^n}{b^n} \\
\sqrt{nm} &= \sqrt{n} \times \sqrt{m} \\
\sqrt\frac{a}{b} &= \frac{\sqrt{a}}{\sqrt{b}} \\
\left( \sqrt[m]a \right)^n &= \sqrt[m]{a^n}
\end{align*}

\section{Арифметическая прогрессия}

\begin{align*}
a_{n} &= a_{n-1} + d \\
a_{n} &= a_{1} + d(n - 1) \\
a_{n} &= \frac{a_{n - 1} + a_{n + 1}}{2} 
\end{align*}

\subsection{Формула разности арифметической прогрессии}

\[ d = a_{n+1} - a_{n} \]

\subsection{Формула суммы n-первых членов арифметической прогрессии}

\begin{align*}
S_{n} &= \frac{2a_{1} + (n - 1)d}{2} \times n \\ \\
S_{n} &= \frac{a_{1} + a_{n}}{2} \times n
\end{align*}

\section{Геометрическая прогрессия}

\begin{align*}
b_{n+1} &= b_{n} \times q, где ~n ~\varepsilon ~N \\
b_{n} &= b_{1} \times q^{n-1} - n-ый ~член ~прогрессии \\
\frac{b_{n}}{b_{n - 1}} &= \frac{b_{n + 1}}{b_{n}}
\end{align*}

\subsection{Сумма n-ых членов}

\begin{align*}
S_{n} &= \frac{(b_{n}q - b_{1})}{q-1} \\ \\
S_{n} &= \frac{b_{1}(q^n - 1)}{q-1}
\end{align*}

% \[  \]

















\end{document} % Конец текста.