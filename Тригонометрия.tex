\documentclass[a4paper,12pt]{article}

\renewcommand{\baselinestretch}{01.80} % расстояние между строками

\usepackage{cmap}					% поиск в PDF
\usepackage{mathtext} 				% русские буквы в формулах
\usepackage[T2A]{fontenc}			% кодировка
\usepackage[utf8]{inputenc}			% кодировка исходного текста
\usepackage[english,russian]{babel}	% локализация и переносы

% Дополнительная работа с математикой
\usepackage{amsmath,amsfonts,amssymb,amsthm,mathtools} % AMS
\usepackage{icomma} % "Умная" запятая

%% Шрифты
\usepackage{euscript}	 % Шрифт Евклид
\usepackage{mathrsfs}    % Красивый матшрифт

%% Перенос знаков в формулах (по Львовскому)
\newcommand*{\hm}[1]{#1\nobreak\discretionary{}
{\hbox{$\mathsurround=0pt #1$}}{}}

% Работа с таблицами
\usepackage{graphicx}
\usepackage[table,xcdraw]{xcolor}

%%% Заголовок
\author{Hun Fauser}
\title{Тригонометрия}
\date{\today}

\begin{document} % Конец преамбулы, начало текста.

\maketitle

\section{Тригонометрические тождества}

\begin{align*}
\sin^2\alpha + \cos^2\alpha &= 1 \\
\tg\alpha &= \frac{\sin\alpha}{\cos\alpha} \\
\ctg\alpha &= \frac{\cos\alpha}{\sin\alpha} \\
\tg\alpha \times \ctg\alpha &= 1 \\
\tg^2\alpha + 1 &= \frac{1}{\cos^2\alpha} \\
\ctg^2\alpha + 1 &= \frac{1}{\sin^2\alpha} \\
\end{align*}

\section{Формулы сложения тригонометричских функций}

\begin{align*}
\cos(\alpha - \beta) &= \cos\alpha \times \cos\beta + \sin\alpha \times \sin\beta \\
\cos(\alpha + \beta) &= \cos\alpha \times \cos\beta - \sin\alpha \times \sin\beta \\
\sin(\alpha - \beta) &= \sin\alpha \times \cos\beta - \cos\alpha \times \sin\beta \\
\sin(\alpha + \beta) &= \sin\alpha \times \cos\beta + \cos\alpha \times \sin\beta \\
\tg(\alpha + \beta) &= \frac{\tg\alpha + \tg\beta}{1 - \tg\alpha \times \tg\beta} \\
\tg(\alpha - \beta) &= \frac{\tg\alpha - \tg\beta}{1 + \tg\alpha \times \tg\beta} \\
\end{align*}

\section{Сумма тригонометричских функций}

\begin{align*}
\sin\alpha + \sin\beta &= 2\sin\frac{a + b}{2} \times \cos\frac{a - b}{2} \\
\sin\alpha - \sin\beta &= 2\sin\frac{a - b}{2} \times \cos\frac{a + b}{2} \\
\cos\alpha + \cos\beta &= 2\cos\frac{a + b}{2} \times \cos\frac{a - b}{2} \\
\cos\alpha - \cos\beta &= -2\sin\frac{a - b}{2} \times \sin\frac{a + b}{2} \\
\end{align*}

\section{Функции кратных углов}

\begin{align*}
\sin2\alpha &= 2\sin\alpha \times \cos\alpha \\
\cos2\alpha &= \cos^2\alpha - \sin^2\alpha \\
\cos2\alpha &= 1 - 2\sin^2\alpha \\
\cos2\alpha &= 2\cos^2\alpha - 1 \\
\tg2\alpha &= \frac{2\tg\alpha}{1 - \tg^2\alpha}
\end{align*}

\section{Еще что-то по тригонометрии}

\begin{align*}
\sin\alpha \times \sin\beta &= \frac{1}{2}(\cos(\alpha - \beta) - \cos(\alpha + \beta)) \\
\cos\alpha \times \cos\beta &= \frac{1}{2}(\cos(\alpha - \beta) + \cos(\alpha + \beta)) \\
\sin\alpha \times \cos\beta &= \frac{1}{2}(\sin(\alpha - \beta) + \sin(\alpha + \beta))
\end{align*}

\section{Формулы понижения степени}

\begin{align*}
\sin^2\alpha &= \frac{1 - \cos2\alpha}{2} \\
\cos^2\alpha &= \frac{1 + \cos2\alpha}{2} \\
\tg^2\alpha &= \frac{1 - \cos2\alpha}{1 + \cos2\alpha} \\
\sin^3\alpha &= \frac{1}{4}(3\sin\alpha - \sin3\alpha) \\
\cos^3\alpha &= \frac{1}{4}(\cos3\alpha + 3\cos\alpha)
\end{align*}

\end{document} % Конец текста.