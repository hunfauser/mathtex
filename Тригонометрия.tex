\documentclass[a4paper,12pt]{article}

\renewcommand{\baselinestretch}{01.8} % расстояние между строками

\usepackage{cmap}					% поиск в PDF
\usepackage{mathtext} 				% русские буквы в формулах
\usepackage[T2A]{fontenc}			% кодировка
\usepackage[utf8]{inputenc}			% кодировка исходного текста
\usepackage[english,russian]{babel}	% локализация и переносы

% Дополнительная работа с математикой
\usepackage{amsmath,amsfonts,amssymb,amsthm,mathtools} % AMS
\usepackage{icomma}                                    % "Умная" запятая

% Шрифты
\usepackage{euscript} % Шрифт Евклид
\usepackage{mathrsfs} % Красивый матшрифт

% Работа с графикой
\usepackage{pgf,tikz} 
\usepackage{pgfplots}
\usepackage{pgfplotstable}

% Заголовок
\author{Hun Fauser}
\title{Тригонометрия}
\date{\today}

\begin{document} % Конец преамбулы, начало текста.

\maketitle

\section{Тригонометрические тождества}

\begin{align*}
\sin^2\alpha + \cos^2\alpha &= 1 \\
\tg\alpha &= \frac{\sin\alpha}{\cos\alpha} \\
\ctg\alpha &= \frac{\cos\alpha}{\sin\alpha} \\
\tg\alpha \times \ctg\alpha &= 1 \\
\tg^2\alpha + 1 &= \frac{1}{\cos^2\alpha} \\
\ctg^2\alpha + 1 &= \frac{1}{\sin^2\alpha} \\
\end{align*}

\section{Формулы сложения тригонометричских функций}

\begin{align*}
\cos(\alpha - \beta) &= \cos\alpha \times \cos\beta + \sin\alpha \times \sin\beta \\
\cos(\alpha + \beta) &= \cos\alpha \times \cos\beta - \sin\alpha \times \sin\beta \\
\sin(\alpha - \beta) &= \sin\alpha \times \cos\beta - \cos\alpha \times \sin\beta \\
\sin(\alpha + \beta) &= \sin\alpha \times \cos\beta + \cos\alpha \times \sin\beta \\
\tg(\alpha + \beta) &= \frac{\tg\alpha + \tg\beta}{1 - \tg\alpha \times \tg\beta} \\
\tg(\alpha - \beta) &= \frac{\tg\alpha - \tg\beta}{1 + \tg\alpha \times \tg\beta} \\
\end{align*}

\section{Сумма тригонометричских функций}

\begin{align*}
\sin\alpha + \sin\beta &= 2\sin\frac{a + b}{2} \times \cos\frac{a - b}{2} \\
\sin\alpha - \sin\beta &= 2\sin\frac{a - b}{2} \times \cos\frac{a + b}{2} \\
\cos\alpha + \cos\beta &= 2\cos\frac{a + b}{2} \times \cos\frac{a - b}{2} \\
\cos\alpha - \cos\beta &= -2\sin\frac{a - b}{2} \times \sin\frac{a + b}{2} \\
\end{align*}

\section{Функции кратных углов}

\begin{align*}
\sin2\alpha &= 2\sin\alpha \times \cos\alpha \\
\cos2\alpha &= \cos^2\alpha - \sin^2\alpha \\
\cos2\alpha &= 1 - 2\sin^2\alpha \\
\cos2\alpha &= 2\cos^2\alpha - 1 \\
\tg2\alpha &= \frac{2\tg\alpha}{1 - \tg^2\alpha}
\end{align*}

\section{Преобразование произведения тригонометрических функций в сумму}

\begin{figure}[h]
\begin{center}
	\begin{tikzpicture}
		\draw(0, 0) circle (1cm);
		\draw(3, 0) circle (1cm);
		\draw(6, 0) circle (1cm);
		
		\draw (0, -1) -- (0, 1);
		\draw (-1, 0) -- (1, 0);
		
		\draw (3, 1) -- (3, -1);
		\draw (2, 0) -- (4, 0);
		
		\draw (6, 1) -- (6, -1);
		\draw (5, 0)-- (7, 0);
		
		\draw [shift={(0, 0)}] plot[domain=0.5:2.6, variable=\t]({1*1.4*cos(\t r)+-0*1.4*sin(\t r)},{0*1.4*cos(\t r)+1*1.4*sin(\t r)});
		\draw (-1.2, 0.73) -- (-0.9, 0.8);
		\draw (-1.2, 0.73) -- (-1.3, 1);
	\begin{scriptsize}
		\draw (-0.4, 0.4) node {$+$};
		\draw (0.4, 0.4) node {$+$};
		\draw (-0.4, -0.4) node {$-$};
		\draw (0.4, -0.4) node {$-$};
		\draw (0, -1.2) node {$\sin$};
		
		\draw (2.6, 0.4) node {$-$};
		\draw (3.4, 0.4) node {$+$};
		\draw (2.6, -0.4) node {$-$};
		\draw (3.4, -0.4) node {$+$};
		\draw (3, -1.2) node {$\cos$};
		
		\draw (5.6, 0.4) node {$-$};
		\draw (6.4, 0.4) node {$+$};
		\draw (5.6, -0.4) node {$+$};
		\draw (6.4, -0.4) node {$-$};
		\draw (6, -1.2) node {$\tg, \ctg$};
	\end{scriptsize}
	\end{tikzpicture}
\end{center}
\end{figure}

\begin{align*}
\sin\alpha \times \sin\beta &= \frac{1}{2}(\cos(\alpha - \beta) - \cos(\alpha + \beta)) \\
\cos\alpha \times \cos\beta &= \frac{1}{2}(\cos(\alpha - \beta) + \cos(\alpha + \beta)) \\
\sin\alpha \times \cos\beta &= \frac{1}{2}(\sin(\alpha - \beta) + \sin(\alpha + \beta))
\end{align*}

\section{Формулы понижения степени}

\begin{align*}
\sin^2\alpha &= \frac{1 - \cos2\alpha}{2} \\
\cos^2\alpha &= \frac{1 + \cos2\alpha}{2} \\
\tg^2\alpha &= \frac{1 - \cos2\alpha}{1 + \cos2\alpha} \\
\sin^3\alpha &= \frac{1}{4}(3\sin\alpha - \sin3\alpha) \\
\cos^3\alpha &= \frac{1}{4}(\cos3\alpha + 3\cos\alpha)
\end{align*}

\end{document} % Конец текста.