\documentclass[a4paper,12pt]{article}

\usepackage{cmap}					% поиск в PDF
\usepackage{mathtext} 				% русские буквы в формулах
\usepackage[T2A]{fontenc}			% кодировка
\usepackage[utf8]{inputenc}			% кодировка исходного текста
\usepackage[english,russian]{babel}	% локализация и переносы

% Дополнительная работа с математикой
\usepackage{amsmath,amsfonts,amssymb,amsthm,mathtools} % AMS
\usepackage{icomma} % "Умная" запятая

%% Шрифты
\usepackage{euscript}	 % Шрифт Евклид
\usepackage{mathrsfs}    % Красивый матшрифт

%% Перенос знаков в формулах (по Львовскому)
\newcommand*{\hm}[1]{#1\nobreak\discretionary{}
{\hbox{$\mathsurround=0pt #1$}}{}}

% Работа с таблицами
\usepackage{graphicx}
\usepackage[table,xcdraw]{xcolor}

%%% Заголовок
\author{Hun Fauser}
\title{Свойства логарифмов}
\date{\today}

\begin{document} % Конец преамбулы, начало текста.

\maketitle

\begin{align*}
\log_{a}1 &= 0 \\
\log_{a}(x_{1} \times x_{2}) &= \log_{a}|x_{1}| + \log_{a}|x_{2}| \\
\log_{a}\frac{x_{1}}{x_{2}} &= \log_{a}|x_{1}| - \log_{a}|x_{2}| \\
\log_{a}x^p &= p\log_{a}x \\
\log_{a}x &= \frac{\log_{b}x}{\log_{b}a} \\
\log_{a^q}x &= \frac{1}{q}\log_{a}x \\
\lg10^n &= n \\
\ln{e^n} &= n
\end{align*}

\end{document} % Конец текста.